\subsection{\XCRYPTO instructions}
\label{sec:spec:instruction}

% =============================================================================

\subsubsection{Class-$1$:   baseline}
\label{sec:spec:instruction:1}


\XCRYPTOINSTR{xc.ldr.bu}{rd, rs1(rs2)}{
  Load a      byte from memory
  into a  $\GPR$ register,
  zero extending to \RVXLEN bits.
  Instruction exists on RV32, RV64 and RV128.
}{
  $addr  \ASN \GPR[*][{\VERB[RV]{rs1}}] + \GPR[*][{\VERB[RV]{rs2}}]$ \;
  $\GPR[*][{\VERB[RV]{rd}}] \ASN \EXT{0}{(\MEM[*][{addr}])}$ \;
}

% -----------------------------------------------------------------------------

\XCRYPTOINSTR{xc.ldr.b}{rd, rs1(rs2)}{
  Load a      byte from memory
  into a  $\GPR$ register,
  sign extending to \RVXLEN bits.
  Instruction exists on RV32, RV64 and RV128.
}{
  $addr  \ASN \GPR[*][{\VERB[RV]{rs1}}] + \GPR[*][{\VERB[RV]{rs2}}]$ \;
  $\GPR[*][{\VERB[RV]{rd}}] \ASN \EXT{\pm}{(\MEM[*][{addr}])}$ \;
}

% -----------------------------------------------------------------------------

\XCRYPTOINSTR{xc.ldr.hu}{rd, rs1(rs2)}{
  Load a  halfword from memory
  into a  $\GPR$ register,
  zero extending to \RVXLEN bits.
  If the effective address $addr$ is not half-word aligned, 
  the memory access {\em must} raise an exception in the same way
  as the base ISA 
  (e.g., per \VERB[RV]{lh}~\cite[Section 2.6]{SCARV:RV:ISA:I:17})
  would.
  Instruction exists on RV32, RV64 and RV128.
}{
  $addr  \ASN \GPR[*][{\VERB[RV]{rs1}}] + (\GPR[*][{\VERB[RV]{rs2}}] \LSH 1)$ \;
  $\GPR[*][{\VERB[RV]{rd}}] \ASN \EXT{0}{(\MEM[*][{addr+1:addr}])}$ \;
}

% -----------------------------------------------------------------------------

\XCRYPTOINSTR{xc.ldr.h}{rd, rs1(rs2)}{
  Load a  halfword from memory
  into a  $\GPR$ register,
  sign extending to \RVXLEN bits.
  If the effective address $addr$ is not half-word aligned, 
  the memory access {\em must} raise an exception in the same way
  as the base ISA 
  (e.g., per \VERB[RV]{lh}~\cite[Section 2.6]{SCARV:RV:ISA:I:17})
  would.
  Instruction exists on RV32, RV64 and RV128.
}{
  $addr  \ASN \GPR[*][{\VERB[RV]{rs1}}] + (\GPR[*][{\VERB[RV]{rs2}}] \LSH 1)$ \;
  $\GPR[*][{\VERB[RV]{rd}}] \ASN \EXT{\pm}{(\MEM[*][{addr+1:addr}])}$ \;
}

% -----------------------------------------------------------------------------

\XCRYPTOINSTR{xc.ldr.wu}{rd, rs1(rs2)}{
  Load a      word from memory 
  into                         an $\GPR$ register,
  zero extending to \RVXLEN bits.
  If the effective address $addr$ is not      word aligned, 
  the memory access {\em must} raise an exception in the same way
  as the base ISA 
  (e.g., per \VERB[RV]{lw}~\cite[Section 2.6]{SCARV:RV:ISA:I:17})
  would.
  Instruction exists on RV64 and RV128.
}{
  $addr  \ASN \GPR[*][{\VERB[RV]{rs1}}] + (\GPR[*][{\VERB[RV]{rs2}}] \LSH 2)$ \;
  $\GPR[*][{\VERB[RV]{rd}}] \ASN \EXT{0}{(\MEM[*][{addr+3:addr}])}$ \;
}

% -----------------------------------------------------------------------------

\XCRYPTOINSTR{xc.ldr.w}{rd, rs1(rs2)}{
  Load a      word from memory 
  into                         an $\GPR$ register,
  sign extending to \RVXLEN bits.
  If the effective address $addr$ is not      word aligned, 
  the memory access {\em must} raise an exception in the same way
  as the base ISA 
  (e.g., per \VERB[RV]{lw}~\cite[Section 2.6]{SCARV:RV:ISA:I:17})
  would.
  Instruction exists on RV32, RV64 and RV128.
}{
  $addr  \ASN \GPR[*][{\VERB[RV]{rs1}}] + (\GPR[*][{\VERB[RV]{rs2}}] \LSH 2)$ \;
  $\GPR[*][{\VERB[RV]{rd}}] \ASN \EXT{\pm}{(\MEM[*][{addr+3:addr}])}$ \;
}

% -----------------------------------------------------------------------------

\XCRYPTOINSTR{xc.ldr.d}{rd, rs1(rs2)}{
  Load a double word from memory 
  into                         an $\GPR$ register,
  sign extending to \RVXLEN bits.
  If the effective address $addr$ is not double word aligned, 
  the memory access {\em must} raise an exception in the same way
  as the base ISA 
  (e.g., per \VERB[RV]{ld}~\cite[Section 5.3]{SCARV:RV:ISA:I:17})
  would.
  Instruction exists on RV64 and RV128.
}{
  $addr  \ASN \GPR[*][{\VERB[RV]{rs1}}] + (\GPR[*][{\VERB[RV]{rs2}}] \LSH 3)$ \;
  $\GPR[*][{\VERB[RV]{rd}}] \ASN \EXT{\pm}{(\MEM[*][{addr+7:addr}])}$ \;
}

% -----------------------------------------------------------------------------

\XCRYPTOINSTR{xc.str.b}{rs3, rs1(rs2)}{
  Store the least significant byte of an $\GPR$ register into memory.
  Instruction exists on RV32, RV64 and RV128.
}{
  $addr  \ASN \GPR[*][{\VERB[RV]{rs1}}] + \GPR[*][{\VERB[RV]{rs2}}]$ \;
  $\MEM[*][{addr  }] \ASN \INDEX[8]{\GPR[*][{\VERB[RV]{rs3}}]}{0}$ \;
}

% -----------------------------------------------------------------------------

\XCRYPTOINSTR{xc.str.h}{rs3, rs1(rs2)}{
  Store the least significant half-word of an $\GPR$ register into memory.
  If the effective address $addr$ is not half-word aligned, 
  the memory access {\em must} raise an exception in the same way
  as the base ISA 
  (e.g., per \VERB[RV]{sh}~\cite[Section 2.6]{SCARV:RV:ISA:I:17})
  would.
  Instruction exists on RV32, RV64 and RV128.
}{
  $addr  \ASN \GPR[*][{\VERB[RV]{rs1}}] + (\GPR[*][{\VERB[RV]{rs2}}] \LSH 1)$ \;
  $\MEM[*][{addr+1:addr}] \ASN \INDEX[16]{\GPR[*][{\VERB[RV]{rs3}}]}{0}$ \;
}

% -----------------------------------------------------------------------------

\XCRYPTOINSTR{xc.str.w}{rs3, rs1(rs2)}{
  Store the least significant word of an $\GPR$ register into memory.
  If the effective address $addr$ is not      word aligned, 
  the memory access {\em must} raise an exception in the same way
  as the base ISA 
  (e.g., per \VERB[RV]{sw}~\cite[Section 2.6]{SCARV:RV:ISA:I:17})
  would.
  Instruction exists on RV32, RV64 and RV128.
}{
  $addr  \ASN \GPR[*][{\VERB[RV]{rs1}}] + (\GPR[*][{\VERB[RV]{rs2}}] \LSH 2)$ \;
  $\MEM[*][{addri+3:addr}] \ASN \INDEX[32]{\GPR[*][{\VERB[RV]{rs3}}]}{0}$ \;
}

% -----------------------------------------------------------------------------

\XCRYPTOINSTR{xc.str.d}{rs3, rs1(rs2)}{
  Store the least significant double word of an $\GPR$ register into memory.
  If the effective address $addr$ is not double word aligned, 
  the memory access {\em must} raise an exception in the same way
  as the base ISA 
  (e.g., per \VERB[RV]{sd}~\cite[Section 5.3]{SCARV:RV:ISA:I:17})
  would.
  Instruction exists on RV64 and RV128.
}{
  $addr  \ASN \GPR[*][{\VERB[RV]{rs1}}] + (\GPR[*][{\VERB[RV]{rs2}}] \LSH 3)$ \;
  $\MEM[*][{addr+7:addr}] \ASN \INDEX[64]{\GPR[*][{\VERB[RV]{rs3}}]}{0}$ \;
}

% =============================================================================

\subsubsection{Class-$2.1$: randomness}
\label{sec:spec:instruction:2:1}

\XCRYPTOINSTR{xc.rngtest}{rd}{
  Inspect the RNG state, applying a suitable ``health check'' on the level 
  of entropy available: the status flag is stored in a $\GPR$ register.
  If, after execution of \VERB[RV]{xc.rngtest},
         $\GPR[*][{\VERB[RV]{rd1}}] = 1$
  (resp. $\GPR[*][{\VERB[RV]{rd1}}] = 0$),
  then the RNG has 
           sufficient 
  (resp. insufficient)
  entropy; the next output sampled via \VERB[RV]{xc.rngsamp} 
  therefore
         can
  (resp. cannot, or at least should not)
  be used.
  Note that the instruction must write either $1$ or $0$ to the
  destination register.
  Writing {\em non-zero} (resp. $1$) makes verification and co-simulation
  much harder.
}{
  $\GPR[*][{\VERB[RV]{rd}}] \TEST{\ASN} \RNG$ \;
}

% -----------------------------------------------------------------------------

\XCRYPTOINSTR{xc.rngseed}{rs1}{
  Inject (or seed) $\RVXLEN$-bits of entropy into the RNG, reading from a
  $\GPR$ register.
}{
  $\RNG \ASN \GPR[*][{\VERB[RV]{rs1}}]$ \;
}

% -----------------------------------------------------------------------------

\XCRYPTOINSTR{xc.rngsamp}{rd}{
  Sample           $\RVXLEN$-bits of entropy from the RNG, writing to   a
  $\GPR$ register.
}{
  $\GPR[*][{\VERB[RV]{rd}}] \ASN \RNG$ \;
}

% =============================================================================

\subsubsection{Class-$2.2$: memory}
\label{sec:spec:instruction:2:2}

The semantics of each instruction in this class could be broadly described 
as involving multiple, potentially non-contiguous memory accesses, i.e., a 
sequence of loads or stores; each such access relates to an effective
address, computed as the sum of 
a) an explicit             base address 
   plus
b) an explicit or implicit offset.
Imagine $n$ memory accesses should be performed by an instruction, using a 
sequence of effective addresses denoted
\[
addr = \LIST{ addr_0, addr_1, \ldots, addr_{n-1} } .
\]
To ensure consistency, the instruction must adhere to the following rules 
(in addition to \REFSEC{sec:spec:exceptions}) wrt. exceptions:

\begin{itemize}

\item {\em All} effective addresses must be checked for misalignment, and
      an exception raised if necessary, before {\em any} associated memory 
      accesses are performed: either
      a) $\exists~i$ for which the effective address $addr_i$ is misaligned, 
         in which case {\em none} of the memory accesses are performed,
         or
      b) $\forall~i$           the effective address $addr_i$ is    aligned, 
         in which case {\em  all} of the memory accesses are performed.
      The rationale for this rule is to prevent partial execution of the
      instruction, and thus partial update of either architectural state
      or the memory hierarchy.

\item {\em Some} memory accesses have the potential to raise an exception
      that cannot be predicted, even with a priori knowledge of $addr$;
      examples include page, protection or bus faults.  In such cases,
      a) update to 
         architectural state 
         {\em is not} permitted,
         whereas
      b) update to the 
         memory hierarchy
         {\em is}     permitted,
         but should be avoided or minimised if possible.
      For example, the partial execution of a load instruction would not
      lead to the destination register being updated; it could, however,
      lead to the state of any cache memory being updated.
      It is imperative that {\em if} micro-architectural state {\em is}
      updated by a faulting instructions, that mitigations against
      attacks which exploit this are taken.

\end{itemize}

% -----------------------------------------------------------------------------

\XCRYPTOINSTR{xc.scatter.b}{rs3, rs1(rs2)}{
  Perform a ``scattered'' store of 
  four      bytes 
  from an $\GPR$ register
  into memory;
  for the $i$-th store, a base-offset style effective address is formed 
  from
                     a  $\GPR$ register (the   base)
  plus
  a      byte within an $\GPR$ register (the offset).

  Only the low $32$-bits of source registers 
  $\VERB[RV]{rs2}$ and $\VERB[RV]{rs3}$
  are ever read.
}{
  $addr_0                                   \ASN \GPR[*][{\VERB[RV]{rs1}}] + \INDEX[ 8]{\GPR[*][{\VERB[RV]{rs2}}]}{0}$ \;
  $addr_1                                   \ASN \GPR[*][{\VERB[RV]{rs1}}] + \INDEX[ 8]{\GPR[*][{\VERB[RV]{rs2}}]}{1}$ \;
  $addr_2                                   \ASN \GPR[*][{\VERB[RV]{rs1}}] + \INDEX[ 8]{\GPR[*][{\VERB[RV]{rs2}}]}{2}$ \;
  $addr_3                                   \ASN \GPR[*][{\VERB[RV]{rs1}}] + \INDEX[ 8]{\GPR[*][{\VERB[RV]{rs2}}]}{3}$ \;
  $\MEM[*][addr_0  ]                        \ASN \INDEX[ 8]{\GPR[*][{\VERB[RV]{rs3}}]}{0}$ \;
  $\MEM[*][addr_1  ]                        \ASN \INDEX[ 8]{\GPR[*][{\VERB[RV]{rs3}}]}{1}$ \;
  $\MEM[*][addr_2  ]                        \ASN \INDEX[ 8]{\GPR[*][{\VERB[RV]{rs3}}]}{2}$ \;
  $\MEM[*][addr_3  ]                        \ASN \INDEX[ 8]{\GPR[*][{\VERB[RV]{rs3}}]}{3}$ \;
}

% -----------------------------------------------------------------------------

\XCRYPTOINSTR{xc.scatter.h}{rs3, rs1(rs2)}{
  Perform a ``scattered'' store of 
  two  half-words
  from an $\GPR$ register
  into memory;
  for the $i$-th store, a base-offset style effective address is formed 
  from
                     a  $\GPR$ register (the   base)
  plus
  a half-word within an $\GPR$ register (the offset).
  Note that the offset amounts from $\VERB[RV]{rs2}$ are automatically
  halfword aligned by the instruction.
  
  Only the low $32$-bits of source registers 
  $\VERB[RV]{rs2}$ and $\VERB[RV]{rs3}$
  are ever read.
}{
  $addr_0                                   \ASN \GPR[*][{\VERB[RV]{rs1}}] + (\INDEX[16]{\GPR[*][{\VERB[RV]{rs2}}]}{0} \LSH 1)$ \;
  $addr_1                                   \ASN \GPR[*][{\VERB[RV]{rs1}}] + (\INDEX[16]{\GPR[*][{\VERB[RV]{rs2}}]}{1} \LSH 1)$ \;
  $\MEM[*][addr_0+1:addr_0]                 \ASN \INDEX[16]{\GPR[*][{\VERB[RV]{rs3}}]}{0}$ \;
  $\MEM[*][addr_1+1:addr_1]                 \ASN \INDEX[16]{\GPR[*][{\VERB[RV]{rs3}}]}{1}$ \;
}

% -----------------------------------------------------------------------------

\XCRYPTOINSTR{xc.gather.b}{rd, rs1(rs2)}{
  Perform a  ``gathered''  load of 
  four      bytes 
  from memory 
  into an $\GPR$ register;
  for the $i$-th  load, a base-offset style effective address is formed 
  from
                     a  $\GPR$ register (the   base)
  plus
  a      byte within an $\GPR$ register (the offset).

  The {\em high} bits of the destination register from $\RVXLEN$ to $32$
  are zeroed.
}{
  $addr_0                                   \ASN \GPR[*][{\VERB[RV]{rs1}}] + \INDEX[ 8]{\GPR[*][{\VERB[RV]{rs2}}]}{0}$ \;
  $addr_1                                   \ASN \GPR[*][{\VERB[RV]{rs1}}] + \INDEX[ 8]{\GPR[*][{\VERB[RV]{rs2}}]}{1}$ \;
  $addr_2                                   \ASN \GPR[*][{\VERB[RV]{rs1}}] + \INDEX[ 8]{\GPR[*][{\VERB[RV]{rs2}}]}{2}$ \;
  $addr_3                                   \ASN \GPR[*][{\VERB[RV]{rs1}}] + \INDEX[ 8]{\GPR[*][{\VERB[RV]{rs2}}]}{3}$ \;
  $\INDEX[ 8]{\GPR[*][{\VERB[RV]{rd}}]}{0} \ASN \MEM[*][addr_0  ]$ \;
  $\INDEX[ 8]{\GPR[*][{\VERB[RV]{rd}}]}{1} \ASN \MEM[*][addr_1  ]$ \;
  $\INDEX[ 8]{\GPR[*][{\VERB[RV]{rd}}]}{2} \ASN \MEM[*][addr_2  ]$ \;
  $\INDEX[ 8]{\GPR[*][{\VERB[RV]{rd}}]}{3} \ASN \MEM[*][addr_3  ]$ \;
}

% -----------------------------------------------------------------------------

\XCRYPTOINSTR{xc.gather.h}{rd, rs1(rs1)}{
  Perform a  ``gathered''  load of 
  two  half-words
  from memory 
  into an $\GPR$ register;
  for the $i$-th  load, a base-offset style effective address is formed 
  from
                     a  $\GPR$ register (the   base)
  plus
  a half-word within an $\GPR$ register (the offset).
  Note that the offset amounts from $\VERB[RV]{rs2}$ are automatically
  halfword aligned by the instruction.

  The {\em high} bits of the destination register from $\RVXLEN$ to $32$
  are zeroed.
}{
  $addr_0                                   \ASN \GPR[*][{\VERB[RV]{rs1}}] + (\INDEX[16]{\GPR[*][{\VERB[RV]{rs2}}]}{0} \LSH 1)$ \;
  $addr_1                                   \ASN \GPR[*][{\VERB[RV]{rs1}}] + (\INDEX[16]{\GPR[*][{\VERB[RV]{rs2}}]}{1} \LSH 1)$ \;
  $\INDEX[16]{\GPR[*][{\VERB[RV]{rd}}]}{0} \ASN \MEM[*][addr_0+1:addr_0]$ \;
  $\INDEX[16]{\GPR[*][{\VERB[RV]{rd}}]}{1} \ASN \MEM[*][addr_1+1:addr_1]$ \;
}

% =============================================================================

\subsubsection{Class-$2.3$: bit-oriented}
\label{sec:spec:instruction:2:3}

% -----------------------------------------------------------------------------

\XCRYPTOINSTR{xc.bop}{rd, rs1, rs2, lut8}{

{\bf TODO:} Sort out how this instruction is encoded.

  Compute each $1$-bit sub-word (i.e.,    bit) of an $\RVXLEN$-bit result
  via application of a function
  \[
  f  : \SET{ 0, 1 }^3 \rightarrow \SET{ 0, 1 }
  \]
  encoded as a truth-table via immediate \VERB[RV]{lut8} st.
  \[
  f( i ) = \INDEX{\VERB[RV]{lut8}}{i}
  \]
  for $0 \leq i <  8$;
  the  input to said function is read    from (the               $1$-bit sub-words of) three $\GPR$ registers,
  and 
  the output                  is written to   (the corresponding $1$-bit sub-word  of) an    $\GPR$ register.
  For example, if
  $\VERB[RV]{lut8} = \RADIX{11110000}{2}$
  then
  $f( \RADIX{000}{2} ) = \RADIX{0}{2}$,
  $f( \RADIX{011}{2} ) = \RADIX{0}{2}$,
  $f( \RADIX{100}{2} ) = \RADIX{1}{2}$,
  and
  $f( \RADIX{111}{2} ) = \RADIX{1}{2}$.
  Note that
  \VERB[RV]{xc.bop}
  allows synthesis of a wider set of functionality;
  for example, it can be used to synthesise 
  standard Boolean instructions in the base ISA
  \[
  \begin{array}{l@{\;}c@{\;}l c l@{\;}c@{\;}l}
  \VERB[RV]{lut8} &=& \RADIX{10001000}{2} &\Longrightarrow& f &=& x \AND y \\
  \VERB[RV]{lut8} &=& \RADIX{11101110}{2} &\Longrightarrow& f &=& x \IOR y \\
  \VERB[RV]{lut8} &=& \RADIX{01100110}{2} &\Longrightarrow& f &=& x \XOR y \\
  \VERB[RV]{lut8} &=& \RADIX{11101000}{2} &\Longrightarrow& f &=& (x \AND y) \IOR ((x \IOR y) \AND z)
  \end{array}
  \]
  by padding 
  \VERB[RV]{lut8} 
  appropriately.
}{
  $f = \VERB[RV]{imm8}$ \;
  $t \ASN 0$ \;
  \For{$i = 0$ {\bf upto} $31$} {
    $\INDEX  {t}{i} \ASN f( \INDEX{\GPR[*][{\VERB[RV]{rd}}]}{i} \CONS \INDEX{\GPR[*][{\VERB[RV]{rs1}}]}{i} \CONS \INDEX{\GPR[*][{\VERB[RV]{rs2}}]}{i} )$ \;
  }
  $\GPR[*][{\VERB[RV]{rd}}] \ASN t$ \;
}

% ----------------------------------------------------------------------------

\XCRYPTOINSTR{xc.lut}{rd, rs1, rs2, rs3}{
  Compute each $4$-bit sub-word (i.e., nibble) of an $\RVXLEN$-bit result
  via application of a function
  \[
  f  : \SET{ 0, 1 }^4 \rightarrow \SET{ 0, 1 }^4
  \]
  encoded st.
  \[
  f( i ) = \INDEX[4]{(\GPR[*][{\VERB[RV]{rs3}}] \CONS \GPR[*][{\VERB[RV]{rs2}}])}{i}
  \]
  for $0 \leq i < 15$;
  the  input to said function is read    from (a                 $4$-bit sub-word of) an     $\GPR$ register,
  and 
  the output                  is written to   (the corresponding $4$-bit sub-word of) an     $\GPR$ register.
  For example, if
  $\GPR[*][{\VERB[RV]{rs3}}] \CONS \GPR[*][{\VERB[RV]{rs2}}] = \RADIX{0123456789ABCDEF}{16}$
  then
  $f( \RADIX{0000}{2} ) = \RADIX{1111}{2}$,
  $f( \RADIX{0001}{2} ) = \RADIX{1110}{2}$,
  $f( \RADIX{1110}{2} ) = \RADIX{0001}{2}$,
  and
  $f( \RADIX{1111}{2} ) = \RADIX{0000}{2}$.
  Note that
  \VERB[RV]{xc.lut}
  allows synthesis of a wider set of functionality;
  for example, it can be used to synthesise 
  the application of a function
  \[
  f' : \SET{ 0, 1 }^2 \rightarrow \SET{ 0, 1 }^2
  \]
  by padding
  $\GPR[*][{\VERB[RV]{rs3}}]$
  and
  $\GPR[*][{\VERB[RV]{rs2}}]$
  appropriately.
}{
  $f = \GPR[*][{\VERB[RV]{rs3}}]_{31\RANGE 0} \CONS \GPR[*][{\VERB[RV]{rs2}}]_{31\RANGE 0}$ \;
  $t \ASN 0$ \;
  \For{$i = 0$ {\bf upto}  $\RVXLEN/4$}{
    $\INDEX[4]{t}{i} \ASN f( \GPR[*][{\VERB[RV]{rs1}}] )$ \;
  }
  $\GPR[*][{\VERB[RV]{rd}}] \ASN t$ \;
}


% =============================================================================

\subsubsection{Class-$2.4$: packed}
\label{sec:spec:instruction:2:4}

Each instruction in this class computes a packed operation: for an operator
$\odot$, register or immediate variants essentially compute

\begin{enumerate}
\item $\INDEX[w]{r}{i} = \INDEX[w]{x}{i} \OP[w][u]{\odot} \INDEX[w]{y}{i}$,
      for $r$, $x$, and $y$ in $\GPR$ registers.
      and
\item $\INDEX[w]{r}{i} = \INDEX[w]{x}{i} \OP[w][u]{\odot}           y    $,
      for $r$       and $x$ in $\GPR$ registers, and an immediate $y$,
\end{enumerate}

\noindent
and for each 
$
0 \leq i < \VERB[RV]{w} ,
$ 
where the field bit width
$
\VERB[RV]{w} \in \SET{ 2, 4, 8, 16}
$ 
is specified in the $\VERB[RV]{pw}$ field of the instruction encoding.
As such, said encoding yields the following cases:

\begin{table}[H]
\centering
\begin{tabular}{@{}ccc@{}}
\toprule
Assembler Mnemonic & {\tt pw} Value & Bit Field Width \\ \midrule
{\tt c }           & \RADIX{00}{2}  & $2 $            \\
{\tt n }           & \RADIX{01}{2}  & $4 $            \\
{\tt b }           & \RADIX{10}{2}  & $8 $            \\
{\tt h }           & \RADIX{11}{2}  & $16$            \\ \bottomrule
\end{tabular}
\end{table}

\noindent As concrete examples:
\begin{itemize}
\item {\tt xc.padd b, c0, c1, c2}
    will add the four pairs of coresponding bytes in c1 and c2.
\item {\tt xc.padd h, c0, c1, c2} 
    will add corresponding halfwords.
\item {\tt xc.psll b, c0, c1, c2} 
    will shift all bytes in
    $\GPR[*][{\VERB[RV]{rs1}}]$
    by the word value in
    $\GPR[*][{\VERB[RV]{rs2}}]$
\item {\tt xc.pror.i n, c0, c1, 3} 
    will rotate right all nibbles in
    $\GPR[*][{\VERB[RV]{rs1}}]$
    by the immediate value.
\end{itemize}

% -----------------------------------------------------------------------------

\XCRYPTOINSTR{xc.padd}{pw, rd, rs1, rs2}{
  Compute a   register-based packed addition, 
  implying $\odot = +$.
}{
  $w = 2^{\VERB[RV]{pw+1}}$ \;
  $n = \RVXLEN / w$\;
  \For{$i = 0$ {\bf upto} $\VERB[RV]{n} - 1$}{
    $\INDEX[w]{\GPR[*][{\VERB[RV]{rd}}]}{i} \ASN \INDEX[w]{\GPR[*][{\VERB[RV]{rs1}}]}{i} \OP[w][u]{+}      \INDEX[w]{\GPR[*][{\VERB[RV]{rs2}}]}{i}$ \;
  }
}

% -----------------------------------------------------------------------------

\XCRYPTOINSTR{xc.psub}{pw, rd, rs1, rs2}{
  Compute a   register-based packed subtraction, 
  implying $\odot = -$.
}{
  $w = 2^{\VERB[RV]{pw+1}}$ \;
  $n = \RVXLEN / w$\;
  \For{$i = 0$ {\bf upto} $\VERB[RV]{n} - 1$}{
    $\INDEX[w]{\GPR[*][{\VERB[RV]{rd}}]}{i} \ASN \INDEX[w]{\GPR[*][{\VERB[RV]{rs1}}]}{i} \OP[w][u]{-}      \INDEX[w]{\GPR[*][{\VERB[RV]{rs2}}]}{i}$ \;
  }
}

% -----------------------------------------------------------------------------

\XCRYPTOINSTR{xc.pmul.l}{pw, rd, rs1, rs2}{
  Compute a   register-based packed multiplication,
  implying $\odot = \times$;
  this instruction captures the  ``low'' half, or $w$ LSBs of 
  each computed product
  (cf. \VERB[RV]{xc.pmul.h},   which captures the ``high'' half, or $w$ MSBs).
}{
  $w = 2^{\VERB[RV]{pw+1}}$ \;
  $n = \RVXLEN / w$\;
  \For{$i = 0$ {\bf upto} $\VERB[RV]{n} - 1$}{
    $t \ASN \INDEX[w]{\GPR[*][{\VERB[RV]{rs1}}]}{i} \OP[w][u]{ \times} \INDEX[w]{\GPR[*][{\VERB[RV]{rs2}}]}{i}$ \;
    $\INDEX[w]{\GPR[*][{\VERB[RV]{rd}}]}{i} \ASN t_{        w - 1 \RANGE 0}$ \;
  }
}

% -----------------------------------------------------------------------------

\XCRYPTOINSTR{xc.pmul.h}{pw, rd, rs1, rs2}{
  Compute a   register-based packed multiplication,
  implying $\odot = \times$;
  this instruction captures the ``high'' half, or $w$ MSBs of 
  each computed product
  (cf. \VERB[RV]{xc.pmul.l},   which captures the  ``low'' half, or $w$ LSBs).
}{
  $w = 2^{\VERB[RV]{pw+1}}$ \;
  $n = \RVXLEN / w$\;
  \For{$i = 0$ {\bf upto} $\VERB[RV]{n} - 1$}{
    $t \ASN \INDEX[w]{\GPR[*][{\VERB[RV]{rs1}}]}{i} \OP[w][u]{ \times} \INDEX[w]{\GPR[*][{\VERB[RV]{rs2}}]}{i}$ \;
    $\INDEX[w]{\GPR[*][{\VERB[RV]{rd}}]}{i} \ASN t_{2 \cdot w - 1 \RANGE w}$ \;
  }
}

% -----------------------------------------------------------------------------

\XCRYPTOINSTR{xc.pclmul.l}{pw, rd, rs1, rs2}{
  Compute a   register-based packed multiplication in $\B{F}_2[\IND{x}]$ (or ``carry-less'' multiplication),
  implying $\odot = \otimes$;
  this instruction captures the  ``low'' half, or $w$ LSBs of 
  each computed product
  (cf. \VERB[RV]{xc.pclmul.h}, which captures the ``high'' half, or $w$ MSBs).
}{
  $w = 2^{\VERB[RV]{pw+1}}$ \;
  $n = \RVXLEN / w$\;
  \For{$i = 0$ {\bf upto} $\VERB[RV]{n} - 1$}{
    $t \ASN \INDEX[w]{\GPR[*][{\VERB[RV]{rs1}}]}{i} \OP[w][u]{\otimes} \INDEX[w]{\GPR[*][{\VERB[RV]{rs2}}]}{i}$ \;
    $\INDEX[w]{\GPR[*][{\VERB[RV]{rd}}]}{i} \ASN t_{        w - 1 \RANGE 0}$ \;
  }
}

% -----------------------------------------------------------------------------

\XCRYPTOINSTR{xc.pclmul.h}{pw, rd, rs1, rs2}{
  Compute a   register-based packed multiplication in $\B{F}_2[\IND{x}]$ (or ``carry-less'' multiplication),
  implying $\odot = \otimes$;
  this instruction captures the ``high'' half, or $w$ MSBs of 
  each computed product
  (cf. \VERB[RV]{xc.pclmul.l}, which captures the  ``low'' half, or $w$ LSBs).
}{
  $w = 2^{\VERB[RV]{pw+1}}$ \;
  $n = \RVXLEN / w$\;
  \For{$i = 0$ {\bf upto} $\VERB[RV]{n} - 1$}{
    $t \ASN \INDEX[w]{\GPR[*][{\VERB[RV]{rs1}}]}{i} \OP[w][u]{\otimes} \INDEX[w]{\GPR[*][{\VERB[RV]{rs2}}]}{i}$ \;
    $\INDEX[w]{\GPR[*][{\VERB[RV]{rd}}]}{i} \ASN t_{2 \cdot w - 1 \RANGE w}$ \;
  }
}

% -----------------------------------------------------------------------------

\XCRYPTOINSTR{xc.psll}{pw, rd, rs1, rs2}{
  Compute a   register-based packed left-shift,
  implying $\odot = \LSH$.
  Keep in mind that
  any out-of-range distance
  $
  \INDEX[w]{\GPR[*][{\VERB[RV]{rs2}}]}{i} \geq w
  $
  will yield
  $
  \INDEX[w]{\GPR[*][{\VERB[RV]{rd}}]}{i} = 0 .
  $
}{
  $w = 2^{\VERB[RV]{pw+1}}$ \;
  $n = \RVXLEN / w$\;
  \For{$i = 0$ {\bf upto} $\VERB[RV]{n} - 1$}{
    $\INDEX[w]{\GPR[*][{\VERB[RV]{rd}}]}{i} \ASN \INDEX[w]{\GPR[*][{\VERB[RV]{rs1}}]}{i} \OP[w][u]{\LSH}             \GPR[*][{\VERB[RV]{rs2}}]    $ \;
  }
}

% -----------------------------------------------------------------------------

\XCRYPTOINSTR{xc.psll.i}{pw, rd, rs1, cshamt}{
  Compute an immediate-based packed left-shift,
  implying $\odot = \LSH$.
  Keep in mind that
  any out-of-range distance
  $
  \VERB[RV]{shamt} \geq w
  $
  will yield
  $
  \INDEX[w]{\GPR[*][{\VERB[RV]{rd}}]}{i} = 0 
  $
}{
  $w = 2^{\VERB[RV]{pw+1}}$ \;
  $n = \RVXLEN / w$\;
  \For{$i = 0$ {\bf upto} $\VERB[RV]{n} - 1$}{
    $\INDEX[w]{\GPR[*][{\VERB[RV]{rd}}]}{i} \ASN \INDEX[w]{\GPR[*][{\VERB[RV]{rs1}}]}{i} \OP[w][u]{\LSH}   \VERB[RV]{cshamt}$ \;
  }
}

% -----------------------------------------------------------------------------

\XCRYPTOINSTR{xc.psrl}{pw, rd, rs1, rs2}{
  Compute a   register-based packed right-shift,
  implying $\odot = \RSH$.
  Keep in mind that
  any out-of-range distance
  $
  \VERB[RV]{shamt} \geq w
  $
  will yield
  $
  \INDEX[w]{\GPR[*][{\VERB[RV]{rd}}]}{i} = 0 .
  $
}{
  $w = 2^{\VERB[RV]{pw+1}}$ \;
  $n = \RVXLEN / w$\;
  \For{$i = 0$ {\bf upto} $\VERB[RV]{n} - 1$}{
    $\INDEX[w]{\GPR[*][{\VERB[RV]{rd}}]}{i} \ASN \INDEX[w]{\GPR[*][{\VERB[RV]{rs1}}]}{i} \OP[w][u]{\RSH}             \GPR[*][{\VERB[RV]{rs2}}]    $ \;
  }
}

% -----------------------------------------------------------------------------

\XCRYPTOINSTR{xc.psrl.i}{pw, rd, rs1, cshamt}{
  Compute an immediate-based packed left-shift,
  implying $\odot = \RSH$.
  Keep in mind that
  any out-of-range distance
  $
  \VERB[RV]{shamt} \geq w
  $
  will yield
  $
  \INDEX[w]{\GPR[*][{\VERB[RV]{rd}}]}{i} = 0 
  $
}{
  $w = 2^{\VERB[RV]{pw+1}}$ \;
  $n = \RVXLEN / w$\;
  \For{$i = 0$ {\bf upto} $\VERB[RV]{n} - 1$}{
    $\INDEX[w]{\GPR[*][{\VERB[RV]{rd}}]}{i} \ASN \INDEX[w]{\GPR[*][{\VERB[RV]{rs1}}]}{i} \OP[w][u]{\RSH}   \VERB[RV]{cshamt}$ \;
  }
}

% -----------------------------------------------------------------------------

\XCRYPTOINSTR{xc.pror}{pw, rd, rs1, rs2}{
  Compute a   register-based packed right-rotate,
  implying $\odot = \RRT$.
  Keep in mind that
  the effective distance will be
  $
  \INDEX[w]{\GPR[*][{\VERB[RV]{rs1}}]}{i} \pmod{w} .
  $
}{
  $w = 2^{\VERB[RV]{pw+1}}$ \;
  $n = \RVXLEN / w$\;
  \For{$i = 0$ {\bf upto} $\VERB[RV]{n} - 1$}{
    $\INDEX[w]{\GPR[*][{\VERB[RV]{rd}}]}{i} \ASN \INDEX[w]{\GPR[*][{\VERB[RV]{rs1}}]}{i} \OP[w][u]{\RRT}             \GPR[*][{\VERB[RV]{rs2}}]    $ \;
  }
}

% -----------------------------------------------------------------------------

\XCRYPTOINSTR{xc.pror.i}{pw, rd, rs1, cshamt}{
  Compute an immediate-based packed right-rotate,
  implying $\odot = \RRT$.
  Keep in mind that
  the effective distance will be
  $
  \VERB[RV]{rtamt} \pmod{w} 
  $
}{
  $w = 2^{\VERB[RV]{pw+1}}$ \;
  $n = \RVXLEN / w$\;
  $\VERB[RV]{rtamt} = \VERB[RV]{cshamt}$\;
  \For{$i = 0$ {\bf upto} $\VERB[RV]{n} - 1$}{
    $\INDEX[w]{\GPR[*][{\VERB[RV]{rd}}]}{i} \ASN \INDEX[w]{\GPR[*][{\VERB[RV]{rs1}}]}{i} \OP[w][u]{\RRT}   \VERB[RV]{rtamt}$ \;
  }
}

% =============================================================================

\subsubsection{Class-$2.5$: multi-precision}
\label{sec:spec:instruction:2:5}

Instructions in this class use upto
$3$ source      $\GPR$ registers
and
$2$ destination $\GPR$ registers,
which demands some compromises wrt. the encoding used.
In short, a compressed specification of the destination $\GPR$ registers is
used, whereby a single encoded \VERB[RV]{rdm} field implies (or maps to) a 
pair 
$
\TUPLE{ \VERB[RV]{rd2}, \VERB[RV]{rd1} }
$
of register addresses st.
a) $\VERB[RV]{rd2} = \VERB[RV]{rd1} + 1$
   (the contiguous'ness rule),
   and
b) $\VERB[RV]{rd2} = 1 \pmod{2}$
   and 
   $\VERB[RV]{rd1} = 0 \pmod{2}$
   (the  odd-even'ness rule).
Concretely then, this means
\[
\begin{array}{l@{\;}c@{\;}r c c@{\;}l@{\;}c@{\;}r@{\;}c@{\;}l@{\;}c@{\;}r@{\;}c}
\VERB[RV]{rdm} &=& \RADIX{0000}{2} &\mapsto& ( & \VERB[RV]{rd2} &=&  1 &,& \VERB[RV]{rd1} &=&  0 & ) \\
\VERB[RV]{rdm} &=& \RADIX{0001}{2} &\mapsto& ( & \VERB[RV]{rd2} &=&  3 &,& \VERB[RV]{rd1} &=&  2 & ) \\
\VERB[RV]{rdm} &=& \RADIX{0010}{2} &\mapsto& ( & \VERB[RV]{rd2} &=&  5 &,& \VERB[RV]{rd1} &=&  4 & ) \\
\VERB[RV]{rdm} &=& \RADIX{0011}{2} &\mapsto& ( & \VERB[RV]{rd2} &=&  7 &,& \VERB[RV]{rd1} &=&  6 & ) \\
\VERB[RV]{rdm} &=& \RADIX{0100}{2} &\mapsto& ( & \VERB[RV]{rd2} &=&  9 &,& \VERB[RV]{rd1} &=&  8 & ) \\
\VERB[RV]{rdm} &=& \RADIX{0101}{2} &\mapsto& ( & \VERB[RV]{rd2} &=& 11 &,& \VERB[RV]{rd1} &=& 10 & ) \\
\VERB[RV]{rdm} &=& \RADIX{0110}{2} &\mapsto& ( & \VERB[RV]{rd2} &=& 13 &,& \VERB[RV]{rd1} &=& 12 & ) \\
\VERB[RV]{rdm} &=& \RADIX{0111}{2} &\mapsto& ( & \VERB[RV]{rd2} &=& 15 &,& \VERB[RV]{rd1} &=& 14 & ) \\
... \\
\VERB[RV]{rdm} &=& \RADIX{1111}{2} &\mapsto& ( & \VERB[RV]{rd2} &=& 31 &,& \VERB[RV]{rd1} &=& 30 & ) \\
\end{array}
\]
i.e.,
\[
\VERB[RV]{rdm} = x ~~\mapsto~~ \TUPLE{ \VERB[RV]{rd2} = x \CONS 1 , \VERB[RV]{rd1} = x \CONS 0 } ,
\]
noting
$\TUPLE{ 1, 0 }$ and $\TUPLE{ 3, 2 }$ are deemed valid pairs,
$\TUPLE{ 3, 0 }$ is invalid because it violates the contiguous'ness rule,
$\TUPLE{ 2, 3 }$ is invalid because it violates the contiguous'ness rule,
and
$\TUPLE{ 2, 1 }$ is invalid because it violates the  odd-even'ness rule.

Equivilent RISC-V code listings for these instructions can be found
in Appendix \REFSEC{sec:design:mpn}.

% -----------------------------------------------------------------------------


\XCRYPTOINSTR{xc.madd.3}{(rd2, rd1), rs1, rs2, rs3}{
  Compute a   register-based multi-precision (or ``long'') addition:
  one can interpret
  $\GPR[*][{\VERB[RV]{rs1}}]$
  and
  $\GPR[*][{\VERB[RV]{rs2}}]$ as the addends,
  $\GPR[*][{\VERB[RV]{rs3}}]$ as the carry-in bit,
  $\GPR[*][{\VERB[RV]{rd1}}]$ as the sum,
  and
  $\GPR[*][{\VERB[RV]{rd2}}]$ as the carry-out,
  meaning the instruction acts as a word-oriented analogue of a full-adder.
  Note that, wrt. the encoding,
  $
  \VERB[RV]{rdm} \mapsto \TUPLE{ \VERB[RV]{rd2}, \VERB[RV]{rd1} } .
  $
}{
  $t                          \ASN ( \GPR[*][{\VERB[RV]{rs1}}] \OP[][u]{+}        \GPR[*][{\VERB[RV]{rs2}}] ) 
                                                               \OP[][u]{+} \INDEX{\GPR[*][{\VERB[RV]{rs3}}]}{0}$ \;
  $\GPR[*][{\VERB[RV]{rd1}}] \ASN \INDEX{t}{      \RVXLEN-1 \RANGE     0}$ \;
  $\GPR[*][{\VERB[RV]{rd2}}] \ASN \INDEX{t}{      \RVXLEN               }$ \;
}


% -----------------------------------------------------------------------------

\XCRYPTOINSTR{xc.msub.3}{(rd2, rd1), rs1, rs2, rs3}{
  Compute a   register-based multi-precision (or ``long'') subtraction:
  one can interpret
  $\GPR[*][{\VERB[RV]{rs1}}]$ as the    minuend,
  $\GPR[*][{\VERB[RV]{rs2}}]$ as the subtrahend,
  $\GPR[*][{\VERB[RV]{rs3}}]$ as the borrow-in bit,
  $\GPR[*][{\VERB[RV]{rd1}}]$ as the difference,
  and
  $\GPR[*][{\VERB[RV]{rd2}}]$ as the borrow-out,
  meaning the instruction acts as a word-oriented analogue of a full-subtractor.
  Note that, wrt. the encoding,
  $
  \VERB[RV]{rdm} \mapsto \TUPLE{ \VERB[RV]{rd2}, \VERB[RV]{rd1} } .
  $
}{
  $t                          \ASN ( \GPR[*][{\VERB[RV]{rs1}}] \OP[][u]{-}        \GPR[*][{\VERB[RV]{rs2}}] ) 
                                                               \OP[][u]{-} \INDEX{\GPR[*][{\VERB[RV]{rs3}}]}{0}  $ \;
  $\GPR[*][{\VERB[RV]{rd1}}] \ASN \INDEX{t}{        \RVXLEN-1 \RANGE       0}$ \;
  $\GPR[*][{\VERB[RV]{rd2}}] \ASN \INDEX{t}{        \RVXLEN                 }$ \;
}

% -----------------------------------------------------------------------------

\XCRYPTOINSTR{xc.macc.1}{(rd2, rd1), rs1, rs2, rs3}{
  Compute a   register-based multi-precision (or ``long'') accumulation,
  with the
  $\RVXLEN$-bit addend
  $\GPR[*][{\VERB[RV]{rs1}}]$
  accumulated into a $( 2 \cdot \RVXLEN )$-bit value formed by concatenating
  $\GPR[*][{\VERB[RV]{rs1}}]$
  and
  $\GPR[*][{\VERB[RV]{rs2}}]$.
  Note that, wrt. the encoding,
  $
  \VERB[RV]{rdm} \mapsto \TUPLE{ \VERB[RV]{rd2}, \VERB[RV]{rd1} } .
  $
}{
  $t                          \ASN ( \GPR[*][{\VERB[RV]{rs1}}] \CONS \GPR[*][{\VERB[RV]{rs2}}] ) \OP[][u]{+} \GPR[*][{\VERB[RV]{rs3}}]$ \;
  $\GPR[*][{\VERB[RV]{rd1}}] \ASN \INDEX{t}{        \RVXLEN-1 \RANGE       0}$ \;
  $\GPR[*][{\VERB[RV]{rd2}}] \ASN \INDEX{t}{2 \cdot \RVXLEN   \RANGE \RVXLEN}$ \;
}

% -----------------------------------------------------------------------------

\XCRYPTOINSTR{xc.mmul.3}{(rd2, rd1), rs1, rs2, rs3}{
  Compute a   register-based multi-precision (or ``long'') multiply-accumulate:
  the $\RVXLEN$-bit multiplicands
  $\GPR[*][{\VERB[RV]{rs1}}]$
  and
  $\GPR[*][{\VERB[RV]{rs2}}]$
  are first used to form a $( 2 \cdot \RVXLEN )$-bit product captured by
  $\GPR[*][{\VERB[RV]{rd1}}]$ (the LSBs)
  and
  $\GPR[*][{\VERB[RV]{rd2}}]$ (the MSBs),
  into which the $\RVXLEN$-bit addend
  $\GPR[*][{\VERB[RV]{rs3}}]$
  is accumulated.
  Note that, wrt. the encoding,
  $
  \VERB[RV]{rdm} \mapsto \TUPLE{ \VERB[RV]{rd2}, \VERB[RV]{rd1} } .
  $
}{
  $t                          \ASN ( \GPR[*][{\VERB[RV]{rs1}}] \OP[][u]{ \times} \GPR[*][{\VERB[RV]{rs2}}] ) 
                                                               \OP[][u]{ +}      \GPR[*][{\VERB[RV]{rs3}}]$ \;
  $\GPR[*][{\VERB[RV]{rd1}}] \ASN \INDEX{t}{        \RVXLEN-1 \RANGE       0}$ \;
  $\GPR[*][{\VERB[RV]{rd2}}] \ASN \INDEX{t}{2 \cdot \RVXLEN   \RANGE \RVXLEN}$ \;
}

\XCRYPTOINSTR{xc.mror}{(rd2, rd1), rs1, rs2, rs3}{
  Compute a register-based double-width right rotate.
}{
  $t \ASN (\GPR[*][{\VERB[RV]{rs1}}] \CONS \GPR[*][{\VERB[RV]{rs2}}]) \RRT \GPR[*][{\VERB[RV]{rs3}}] $\;
  $\GPR[*][{\VERB[RV]{rd1}}] \ASN \INDEX{t}{        \RVXLEN-1 \RANGE       0}$ \;
  $\GPR[*][{\VERB[RV]{rd2}}] \ASN \INDEX{t}{2 \cdot \RVXLEN   \RANGE \RVXLEN}$ \;
}

% -----------------------------------------------------------------------------


\subsubsection{Class-$3.1$: AES}
\label{sec:spec:instruction:3:1}

\XCRYPTOINSTR{xc.aessub}{.\[enc,dec\]\[rot\] rd, rs1, rs2}{
  Per~\cite[Figures 5]{SCARV:TilGro:06}, 
  apply a (carefully structured) form of the AES 
  {\tt SubBytes}~\cite[Section 5.1.1]{SCARV:FIPS:197}
  operation:
  this is applicable to a column-oriented implementation of AES, where
  state (resp. round key) matrix columns are packed into $32$-bit
  words.
  Note that, wrt. the encoding,
  $
  \VERB[RV]{mode}
  $
  and
  $
  \VERB[RV]{rot}
  $
  control  which variant is applied: these variants are exposed via
  dedicated mnemonics st.
  \[
  \begin{array}{l c r@{\;}c@{\;}l @{\;}c@{\;} r@{\;}c@{\;}l}
  \VERB[RV]{xc.aessub.enc}    &\mapsto& \VERB[RV]{mode} &=& 0 &,& \VERB[RV]{rot} &=& 0 \\
  \VERB[RV]{xc.aessub.encrot} &\mapsto& \VERB[RV]{mode} &=& 0 &,& \VERB[RV]{rot} &=& 1 \\
  \VERB[RV]{xc.aessub.dec}    &\mapsto& \VERB[RV]{mode} &=& 1 &,& \VERB[RV]{rot} &=& 0 \\
  \VERB[RV]{xc.aessub.decrot} &\mapsto& \VERB[RV]{mode} &=& 1 &,& \VERB[RV]{rot} &=& 1 \\
  \end{array}
  \]
}{
  \eIf{$\VERB[RV]{mode} = 0$}{
    $t_0 \ASN \SCOPE{\ID{AES}}{\ALG{S-box}}     ( \GPR[*][{\VERB[RV]{rs1}}]_{ 7 \RANGE  0} )$ \;
    $t_1 \ASN \SCOPE{\ID{AES}}{\ALG{S-box}}     ( \GPR[*][{\VERB[RV]{rs2}}]_{15 \RANGE  8} )$ \;
    $t_2 \ASN \SCOPE{\ID{AES}}{\ALG{S-box}}     ( \GPR[*][{\VERB[RV]{rs1}}]_{23 \RANGE 16} )$ \;
    $t_3 \ASN \SCOPE{\ID{AES}}{\ALG{S-box}}     ( \GPR[*][{\VERB[RV]{rs2}}]_{31 \RANGE 24} )$ \;
  }{
    $t_0 \ASN \SCOPE{\ID{AES}}{\ALG{S-box}}^{-1}( \GPR[*][{\VERB[RV]{rs1}}]_{ 7 \RANGE  0} )$ \;
    $t_1 \ASN \SCOPE{\ID{AES}}{\ALG{S-box}}^{-1}( \GPR[*][{\VERB[RV]{rs2}}]_{15 \RANGE  8} )$ \;
    $t_2 \ASN \SCOPE{\ID{AES}}{\ALG{S-box}}^{-1}( \GPR[*][{\VERB[RV]{rs1}}]_{23 \RANGE 16} )$ \;
    $t_3 \ASN \SCOPE{\ID{AES}}{\ALG{S-box}}^{-1}( \GPR[*][{\VERB[RV]{rs2}}]_{31 \RANGE 24} )$ \;
  }
  
  \eIf{$\VERB[RV]{rot}  = 0$}{
    $\GPR[*][{\VERB[RV]{rd}}] \ASN \EXT{0}{(t_3 \CONS t_2 \CONS t_1 \CONS t_0)} $ \;
  }{
    $\GPR[*][{\VERB[RV]{rd}}] \ASN \EXT{0}{(t_2 \CONS t_1 \CONS t_0 \CONS t_3)} $ \;
  }
}

\XCRYPTOINSTR{xc.aesmix}{mode,      rd, rs1, rs2}{
  Per~\cite[Figures 5]{SCARV:TilGro:06}, 
  apply a (carefully structured) form of the AES 
  {\tt ShiftRows}~\cite[Section 5.1.2]{SCARV:FIPS:197} 
  and
  {\tt MixColumns}~\cite[Section 5.1.3]{SCARV:FIPS:197}
  operations:
  this is applicable to a column-oriented implementation of AES, where
  state (resp. round key) matrix columns are packed into $32$-bit
  words.
  Note that, wrt. the encoding,
  $
  \VERB[RV]{mode}
  $
  controls which variant is applied: these variants are exposed via
  dedicated mnemonics st.
  \[
  \begin{array}{l c r@{\;}c@{\;}l@{\;}}
  \VERB[RV]{xc.aesmix.enc}    &\mapsto& \VERB[RV]{mode} &=& 0 \\
  \VERB[RV]{xc.aesmix.dec}    &\mapsto& \VERB[RV]{mode} &=& 1 \\
  \end{array}
  \]
}{
    $t_0 \ASN \GPR[*][{\VERB[RV]{rs1}}]_{ 7 \RANGE  0}  $ \;
    $t_1 \ASN \GPR[*][{\VERB[RV]{rs1}}]_{15 \RANGE  8}  $ \;
    $t_2 \ASN \GPR[*][{\VERB[RV]{rs2}}]_{23 \RANGE 16}  $ \;
    $t_3 \ASN \GPR[*][{\VERB[RV]{rs2}}]_{31 \RANGE 24}  $ \;

  \eIf{$\VERB[RV]{mode} = 0$}{
    $result \ASN \SCOPE{\ID{AES}}{\ALG{Mix-Column}}     ( t_3 \CONS t_2 \CONS t_1 \CONS t_0 )$ \;
  }{
    $result \ASN \SCOPE{\ID{AES}}{\ALG{Mix-Column}}^{-1}( t_3 \CONS t_2 \CONS t_1 \CONS t_0 )$ \;
  }
  $\GPR[*][{\VERB[RV]{rd}}] \ASN \EXT{0}{(result)}$\;
}

% -----------------------------------------------------------------------------


\subsubsection{Class-$3.2$: SHA2}
\label{sec:spec:instruction:3:2}

These instructions are designed to accelerate critical parts of the
SHA2 round functions.

The SHA256 instructions are available on RV32, RV64 and RV128.
The SHA512 instructions are available on       RV64 and RV128.

The exact part of the round function computed is exposed by dedicated
mnemonics st.
  \[
  \begin{array}{l c r@{\;}c@{\;}l@{\;}}
  \VERB[RV]{xc.sha256.s0}, \VERB[RV]{xc.sha512.s0} &\mapsto& \VERB[RV]{s} &=& 0 \\
  \VERB[RV]{xc.sha256.s1}, \VERB[RV]{xc.sha512.s1} &\mapsto& \VERB[RV]{s} &=& 1 \\
  \VERB[RV]{xc.sha256.s2}, \VERB[RV]{xc.sha512.s2} &\mapsto& \VERB[RV]{s} &=& 2 \\
  \VERB[RV]{xc.sha256.s3}, \VERB[RV]{xc.sha512.s3} &\mapsto& \VERB[RV]{s} &=& 3 \\
  \end{array}
  \]

Some design rationale for these instructions can be found
in Appendix \REFSEC{sec:design:sha}.

\XCRYPTOINSTR{xc.sha256}{.s\[0,1,2,3\] rd, rs1}{
Compute parts of the SHA256 round function.
Each operation works on $32$-bit variables and intermediate values.
The result is zero-extended to \RVXLEN bits before being written back
to $\GPR[*][{\VERB[RV]{rd}}]$.

These instructions are available on RV32, RV64 and RV128.
}{
    $a \ASN \GPR[*][{\VERB[RV]{rs1}}]_{31 \RANGE 0}$\;

    \uIf{$s = 0$}{
        $result \ASN (a \RRT  7) \XOR (a \RRT 18) \XOR (a \RSH  3)$ \;
    }\uElseIf{$s = 1$}{
        $result \ASN (a \RRT 17) \XOR (a \RRT 19) \XOR (a \RSH 10)$ \;
    }\uElseIf{$s = 2$}{
        $result \ASN (a \RRT  2) \XOR (a \RRT 13) \XOR (a \RRT 22)$ \;
    }\uElseIf{$s = 3$}{
        $result \ASN (a \RRT  6) \XOR (a \RRT 11) \XOR (a \RRT 25)$ \;
    }

    $\GPR[*][{\VERB[RV]{rd}}] \ASN \EXT{0}{(result)}$\;
}

\XCRYPTOINSTR{xc.sha512}{.s\[0,1,2,3\] rd, rs1}{
Compute parts of the SHA512 round function.
Each operation works on $64$-bit variables and intermediate values.
The result is zero-extended to \RVXLEN bits before being written back
to $\GPR[*][{\VERB[RV]{rd}}]$.

These instructions are available on       RV64 and RV128.
}{
    $a \ASN \GPR[*][{\VERB[RV]{rs1}}]_{63 \RANGE 0}$\;

    \uIf{$s = 0$}{
        $result \ASN (a \RRT  1) \XOR (a \RRT  8) \XOR (a \RSH  7)$ \;
    }\uElseIf{$s = 1$}{
        $result \ASN (a \RRT 19) \XOR (a \RRT 61) \XOR (a \RSH  6)$ \;
    }\uElseIf{$s = 2$}{
        $result \ASN (a \RRT 28) \XOR (a \RRT 34) \XOR (a \RRT 39)$ \;
    }\uElseIf{$s = 3$}{
        $result \ASN (a \RRT 14) \XOR (a \RRT 18) \XOR (a \RRT 41)$ \;
    }

    $\GPR[*][{\VERB[RV]{rd}}] \ASN \EXT{0}{(result)}$\;
}

% ----------------------------------------------------------------------------

\subsubsection{Class-$3.3$: SHA3}
\label{sec:spec:instruction:3:3}

Each instruction in this class computes an index into the SHA3 state array:
they provide domain-specific support for address generation within the step
mappings~\cite[Section 3.2]{SCARV:FIPS:202} (i.e., $\Omega$, $\rho$, $\pi$, 
$\chi$, and $\iota$) that constitute each round.

Note that the $2$-bit immediate \VERB[RV]{shamt} can be used align the index
to a byte, half-word, word, or double-word boundary (and thereby support all
SHA3 parameterisations), and

Some design rationale for these instructions can be found
in Appendix \REFSEC{sec:design:sha}.

\XCRYPTOINSTR{xc.sha3.xy}{rd, rs1, rs2, shamt}{
  Compute an index into the SHA3 state array.
}{
  $x                        \ASN \GPR[*][{\VERB[RV]{rs1}}]_{2 \RANGE 0}$ \;
  $y                        \ASN \GPR[*][{\VERB[RV]{rs2}}]_{2 \RANGE 0}$ \;
  $\GPR[*][{\VERB[RV]{rd}}] \ASN ( (   x       \bmod 5 ) + 5 \cdot (                       y   \bmod 5 ) ) \LSH \VERB[RV]{shamt}$ \;
}

\XCRYPTOINSTR{xc.sha3.x1}{rd, rs1, rs2, shamt}{
  Compute an index into the SHA3 state array.
}{
  $x                        \ASN \GPR[*][{\VERB[RV]{rs1}}]_{2 \RANGE 0}$ \;
  $y                        \ASN \GPR[*][{\VERB[RV]{rs2}}]_{2 \RANGE 0}$ \;
  $\GPR[*][{\VERB[RV]{rd}}] \ASN ( ( ( x + 1 ) \bmod 5 ) + 5 \cdot (                       y   \bmod 5 ) ) \LSH \VERB[RV]{shamt}$ \;
}

\XCRYPTOINSTR{xc.sha3.x2}{rd, rs1, rs2, shamt}{
  Compute an index into the SHA3 state array.
}{
  $x                        \ASN \GPR[*][{\VERB[RV]{rs1}}]_{2 \RANGE 0}$ \;
  $y                        \ASN \GPR[*][{\VERB[RV]{rs2}}]_{2 \RANGE 0}$ \;
  $\GPR[*][{\VERB[RV]{rd}}] \ASN ( ( ( x + 2 ) \bmod 5 ) + 5 \cdot (                       y   \bmod 5 ) ) \LSH \VERB[RV]{shamt}$ \;
}

\XCRYPTOINSTR{xc.sha3.x4}{rd, rs1, rs2, shamt}{
  Compute an index into the SHA3 state array.
}{
  $x                        \ASN \GPR[*][{\VERB[RV]{rs1}}]_{2 \RANGE 0}$ \;
  $y                        \ASN \GPR[*][{\VERB[RV]{rs2}}]_{2 \RANGE 0}$ \;
  $\GPR[*][{\VERB[RV]{rd}}] \ASN ( ( ( x + 4 ) \bmod 5 ) + 5 \cdot (                       y   \bmod 5 ) ) \LSH \VERB[RV]{shamt}$ \;
}

\XCRYPTOINSTR{xc.sha3.yx}{rd, rs1, rs2, shamt}{
  Compute an index into the SHA3 state array.
}{
  $x                        \ASN \GPR[*][{\VERB[RV]{rs1}}]_{2 \RANGE 0}$ \;
  $y                        \ASN \GPR[*][{\VERB[RV]{rs2}}]_{2 \RANGE 0}$ \;
  $\GPR[*][{\VERB[RV]{rd}}] \ASN ( (   y       \bmod 5 ) + 5 \cdot ( ( 2 \cdot x + 3 \cdot y ) \bmod 5 ) ) \LSH \VERB[RV]{shamt}$ \;
}

% =============================================================================
