% =============================================================================

\subsection{Concept}
\label{sec:bg:concept}

\begin{itemize}

\item Assuming carte blanche wrt. design and implementation of a processor 
      core, and the wider system it is coupled to, a large design space of 
      candidate approaches exists.  One could, for example, realise a given 
      cryptographic primitive using
      a) hardware-only,
      b) mixed (or hybrid),
         or
      c) software-only
      techniques.

      The remit of \XCRYPTO is strictly focused on an approach that supports the
      latter techniques; one could view said remit as focused on maximising
      the viability of software-based cryptographic implementations.  It has
      an implicit focus on constrained, e.g., micro-controller class, cores, 
      although the concept and utility is broader.
      Various published work is emerging that motivates and/or evidences such
      an approach: 
      see, e.g.,~\cite{SCARV:Stoffelen:19}.

\item The following design criteria rationalise this approach, some of which
      are related (or even stem directly from one and other):

      \begin{itemize}
      \item {\bf         Security}.
            An unfortunate fact is that security will commonly be relegated 
            to a second-class design metric, and so, by implication, viewed 
            as being of secondary importance;
            see, e.g., \cite{SCARV:Lee:03,SCARV:RKLMR:03,SCARV:RRKH:04,SCARV:BurMutTiw:16}.
            In contrast, \XCRYPTO treats security as a first-class metric, and
            so {\em at least} as important as more traditional alternatives.
            For example,
            within the context of \XCRYPTO we deem it reasonable to trade-off 
            improved security vs. degradation of instruction throughput.
      \item {\bf      Consistency}.
            As far as is reasonable, \XCRYPTO should remain consistent with the
            RISC-V philosophy, and associated base ISA; doing so will demand 
            considered compromises vs. a clean-slate design, but, equally,
            should maximise the resulting utility.
            For example,
            we attempt to avoid or minimise 
            a) deviation from the existing instruction encoding formats,
               and 
            b) introduction of additional state.
      \item {\bf       Generality}.
            \XCRYPTO should be a general-purpose cryptographic ISE: bar certain
            controlled (i.e., well motivated) instances, it avoids including 
            (overly) functionality-specific features.
      \item {\bf      Flexibility}. 
            Modulo the status of security as a first-class design metric, a
            goal of \XCRYPTO remains to avoid, or at least minimise any ``baked 
            in'' (or hard coded) trade-offs.  Put another way, although the 
            approach used clearly {\em will} imply trade-offs 
            (e.g., vs. efficiency typically delivered by hardware-only techniques), 
            it {\em should} more easily support
            a) agility wrt. primitive, algorithm, and parameter choices,
               and
            b) instrumentation of context-dependent countermeasures.
      \item {\bf    Composability}.
            To mitigate a given attack, a layered approach (cf. defence in 
            depth) is normally preferred: this favours the use of multiple
            countermeasures, vs. a single, perfect panacea.  The same form
            of argument applies to efficiency, in the sense that efficiency 
            requirements may render software-only implementation techniques,
            even {\em with} support of \XCRYPTO, insufficient.
            As such, \XCRYPTO
            a) should be viewed as one option for or layer in a solution,
               and
            b) prefers features that can co-exist over those which cannot.
            For example, 
            Grabher et al.~\cite{SCARV:GGHJPTW:11} have explored the use of 
            an embedded FPGA fabric\footnote{
            See, e.g., \url{http://www.flex-logix.com}
            } to support cryptographic workloads: \XCRYPTO composes with such 
            an approach.
      \item {\bf Implementability}. 
            To be useful, it should be possible to implement \XCRYPTO with as
            little
            a) overhead   (e.g., wrt. additional logic),
               and
            b) difficulty (e.g., wrt. outright complexity, or complicating factors such as verification)
            as possible.  This implies a preference for features that avoid 
            a {\em requirement} for complex hardware or invasive alteration 
            to the host core.
      \item {\bf    Measurability}.
            Given the remit of extending the base ISA, any feature in \XCRYPTO 
            should offer reproducible, demonstrable value vs. this baseline.
            For a given feature, this goal should be supported by provision 
            of associated reference implementations of cryptographically 
            interesting benchmark kernels.
      \end{itemize}

\item Aligning with several cited design criteria, \XCRYPTO can be viewed as a 
      form of {\em meta}-extension in the sense it captures feature classes 
      that can be selected from to suit.  \REFSEC{sec:bg:feature} outlines 
      said feature classes, the presence or absence of which is reflected 
      in an associated CSR outlined in \REFSEC{sec:spec:state:csr}.

\item All that said, such an approach may naturally be deemed inappropriate 
      for some use-cases due to the trade-offs implied; we carefully pitch 
      \XCRYPTO as {\em an} approach, not {\em the} approach.  
      As such, instances of alternative approaches will naturally exist and
      be perfectly valid.  For example,
      a) the RISC-V Cryptographic Extensions Task Group is focused on use 
         of the vector ISE as a vehicle for cryptographic operations,
      b) many commercialised SoCs 
         (e.g., Google Titan) 
         prefer to couple a dedicated, special-purpose IP core 
         (e.g., a hardware-based AES accelerator) 
         to a RISC-V host core.

\end{itemize}

% =============================================================================
