% =============================================================================

\subsection{Concept}
\label{sec:bg:concept}

\begin{itemize}

\item Assuming carte blanche wrt. design and implementation of a processor 
      core, and the wider system it is coupled to, a large design space of 
      candidate approaches exists.  One could, for example, realise a given 
      cryptographic primitive using
      a) hardware-only,
      b) mixed (or hybrid),
         or
      c) software-only
      techniques.
      The remit of \XCID is strictly focused on an approach that supports the
      latter techniques; one could view said remit as focused on maximising
      the viability of software-based cryptographic implementations.

% TODO: Link to appendix FAQ on general v.s. special purpose ISE meaning.
\item \XCID is best described as a general-purpose cryptographic ISE, and so
      specified along those lines.  However, the implementation of \XCID is
      left more open-ended: one could consider approaches including
      a) a highly   invasive extension to an existing host core,
      b) a somewhat invasive, tightly-coupled co-processor,
         or
      c) a      non-invasive, loosely-coupled co-processor.

\item All cited implementation approaches are viable, but, in {\em concept},
      we view \XCID as a co-processor technology: wlog. we adopt associated
      terminology from here onward.  Phrased as such, \XCID has an implicit
      focus on micro-controller class host cores, although the concept and 
      utility is broader.

      By adopting this perspective, we aim to present a clean(er) separation 
      between the intended roles of the host core and \XCID co-processor.
      For example, at a high level one could view the co-processor as being 
      tasked with support for execution of cryptography; such a task is then
      not expected of the host core.  Along the same lines, the host core 
      would be tasked with orchestrating control-flow, off-loading computation 
      to the co-processor: wrt. cryptographic workloads, one could view the
      former as a general-purpose control-path for a (more) special-purpose 
      data-path represented by the latter.  Finally, note that considering
      physically separate (or at least separatable) implementations offers
      the potential to use differing technologies; one motivation for this 
      could be the (selected) use of a secure logic style 
      (see, e.g.,~\cite{SCARV:MayMur:16})
      and thus design concepts such as PowerTrust~\cite{SCARV:TilKirSze:10}.

\item Consider, for example and in a general sense, ARM {\sf big.LITTLE}
      (an instance of the general Asymmetric Multi-core Processor (AMP)~\cite{SCARV:Mittal:16} concept),
      which can be viewed as combining and switching between
      use of
      a ``big''    core (which delivers efficiency wrt. instruction throughput)
      and 
      a ``little'' core (which delivers efficiency wrt. energy consumption).
      The underlying goal is to capitalise on the characteristics of each as 
      and when need be (at run-time), vs. a single core which represents a 
      (design-time) compromise between those characteristics.  Along similar 
      lines, one could imagine some form of {\sf efficient.SECURE} analogue: 
      we have a 
      an ``efficient'' core (which ignores   security, focusing on delivery of efficiency)
      and
      a  ``secure''    core (which ignores efficiency, focusing on delivery of   security),
      instantiated by the host core and \XCID co-processor respectively.

\item The following design criteria rationalise this approach, some of which
      are related (or even stem directly from one and other):

      \begin{itemize}
      \item {\bf         Security}.
            An unfortunate fact is that security will commonly be relegated 
            to a second-class design metric, and so, by implication, viewed 
            as being of secondary importance;
            see, e.g., \cite{SCARV:Lee:03,SCARV:RKLMR:03,SCARV:RRKH:04,SCARV:BurMutTiw:16}.
            In contrast, \XCID treats security as a first-class metric, and
            so {\em at least} as important as more traditional alternatives.
            For example,
            within the context of \XCID we deem it reasonable to trade-off 
            improved security vs. degradation of instruction throughput.
      \item {\bf      Consistency}.
            As far as is reasonable, \XCID should remain consistent with the
            RISC-V philosophy, and associated base ISA; doing so will demand 
            considered compromises vs. a clean-slate design, but, equally,
            should maximise the resulting utility.
            For example,
            we attempt to avoid or minimise 
            a) deviation from the existing instruction encoding formats,
               and 
            b) introduction of additional state.
            % TODO: "Where we have deviated from RISC-V norms, we have
            %        justified these decisions in Appendix X"
      \item {\bf       Generality}.
            Bar some controlled instances, \XCID endeavours to avoid including
            any (overly) functionality-specific features (and, therefore, any
            associated hardware).
            % TODO: again, "where we have included niche or specific
            %       functionality, we justify it accordingly..."
      \item {\bf      Flexibility}. 
            Modulo the status of security as a first-class design metric, a
            goal of \XCID remains to avoid, or at least minimise any ``baked 
            in'' (or hard coded) trade-offs.  Put another way, although the 
            approach used clearly {\em will} imply trade-offs 
            (e.g., vs. efficiency typically delivered by hardware-only techniques), 
            it {\em should} more easily support
            a) agility wrt. primitive, algorithm, and parameter choices,
               and
            b) instrumentation of context-dependent countermeasures.
      \item {\bf    Composability}.
            To mitigate a given attack, a layered approach (cf. defence in 
            depth) is normally preferred: this favours the use of multiple
            countermeasures, vs. a single, perfect panacea.  The same form
            of argument applies to efficiency, in the sense that efficiency 
            requirements may render software-only implementation techniques,
            even {\em with} support of \XCID, insufficient.
            As such, \XCID
            a) should be viewed as one option for or layer in a solution,
               and
            b) prefers features that can co-exist over those which cannot.
            For example, 
            Grabher et al.~\cite{SCARV:GGHJPTW:11} have explored the use of 
            an embedded FPGA fabric\footnote{
            See, e.g., \url{http://www.flex-logix.com}
            } to support cryptographic workloads: \XCID composes with such 
            an approach.
      \item {\bf Implementability}. 
            To be useful, it should be possible to implement \XCID with as
            little
            a) overhead   (e.g., wrt. additional logic),
               and
            b) difficulty (e.g., wrt. outright complexity, or complicating factors such as verification)
            as possible.  This implies a preference for features that avoid 
            a {\em requirement} for complex hardware or invasive alteration 
            to the host core.
      \item {\bf    Measurability}.
            Given the remit of extending the base ISA, any feature in \XCID 
            should offer reproducible, demonstrable value vs. this baseline.
            For a given feature, this goal should be supported by provision 
            of associated reference implementations of cryptographically 
            interesting benchmark kernels.
      \end{itemize}

\item Aligning with several cited design criteria, \XCID can be viewed as a
      form of {\em meta}-extension in the sense it captures feature classes
      that can be selected from to suit.  \REFSEC{sec:bg:feature} outlines 
      said feature classes, the presence or absence of which is reflected 
      in an associated CSR outlined in \REFSEC{sec:spec:state:csr}.

\item All that said, such an approach may naturally be deemed inappropriate 
      for some use-cases; we carefully pitch \XCID as {\em an} approach, not 
      {\em the} approach.  
      Along these lines, note that instances of the {\em opposite} approach
      naturally exist as alternatives.  For example,
      a) the RISC-V Cryptographic Extensions Task Group is focused on use 
         of the vector ISE as a vehicle for cryptographic operations,
      b) many commercialised SoCs 
         (e.g., Google Titan, Rambus CryptoManager) 
         prefer to couple a dedicated, special-purpose IP core 
         (e.g., a hardware-based AES accelerator) 
         to a RISC-V host core.

\end{itemize}

% =============================================================================
