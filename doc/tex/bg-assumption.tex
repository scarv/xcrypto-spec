% =============================================================================

\subsection{Assumptions}
\label{sec:bg:assumption}

\begin{itemize}

\item Since \XCID is an explicitly an extension of RV32I, we retain the use
      of \RVXLEN as a general placeholder but assume $\RVXLEN = 32$.

\item \XCID demands interaction with an RNG, the concrete instantiation of 
      which is unspecified: we assume the RNG design follows best-practice,
      e.g., per NIST~\cite{SCARV:NIST:SP:800_90a,SCARV:NIST:SP:800_90b,SCARV:NIST:SP:800_90c},
      and has an interface per \cite[Section 6.4]{SCARV:NIST:SP:800_90c}.

      On one hand, doing so affords flexibility in an implementation; this 
      is important, in that an RNG selection and implementation will likely 
      be technology-specific (e.g., differ for a given FPGA, vs. an ASIC).  
      On the other hand, however, the RNG used is critically important wrt. 
      security: the (difficult) challenge of selecting and implementing 
      a robust RNG instance is assumed to be addressed.

\item \XCID assumes a byte-addressable memory, the interface to which will
      be shared between the host core and co-processor.  As such, {\em all}
      attack vectors
      (see, e.g.,~\cite{SCARV:GYCH:18})
      that (ab)use the memory interface {\em must} be robustly mitigated, 
      as would be the case without \XCID (i.e., using the host core alone).

\end{itemize}

% =============================================================================
