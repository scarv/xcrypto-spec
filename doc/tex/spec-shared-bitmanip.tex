\subsection{Shared Bitmanip instructions}
\label{sec:spec:shared:bitmanip}

In the Bitmanip extension, instructions for permutation, field
insertion/extraction and conditional move operations are very
useful for various cryptographic workloads.

Here is listed a set of instructions which are {\em shared} with
the bitmanip extensions.
That is, someone implementing either the \XCID specifcation, or
the Bitmanip extension can use them.
The instruction functions and encodings remain identical across extensions.

For a full description of the functionality of these instructions, we
refer readers to the Bitmanip draft standard\footnote{
\url{https://github.com/riscv/riscv-bitmanip}
}.

\subsubsection{Conditional Move}
The conditional move instruction allows a destination register to
be written with one of two source registers, based on whether a
third source register is zero.
Conditional moves are very useful for implementing constant time
cryptographic code.

Instruction Listing:
\begin{itemize}
\item {\tt cmov rd, rs2, rs1, rs3}
\end{itemize}

\subsubsection{Rotation}
Rotation is an essential primitive operation in many block ciphers and
hash functions.

Instruction Listing:
\begin{itemize}
\item RV32, RV64:
\begin{itemize}
\item {\tt ror  rd, rs1, rs2}
\item {\tt rol  rd, rs1, rs2}
\item {\tt rori rd, rs1, imm}
\end{itemize}
\item RV64 only:
\begin{itemize}
\item {\tt rorw  rd, rs1, rs2}
\item {\tt rolw  rd, rs1, rs2}
\item {\tt roriw rd, rs1, imm}
\end{itemize}
\end{itemize}

\subsubsection{Funnel Shift}
Funnel shifts help with implementing $2*\RVXLEN$ rotations.
This can become very common in RV32I systems executing crypographic code.
Examples include some blockciphers (Prince) and
hash functions (SHA3, SHA512, Blake2).

Instruction Listing:
\begin{itemize}
\item RV32, RV64:
\begin{itemize}
\item {\tt fsl  rd, rs1, rs3, rs2}
\item {\tt fsr  rd, rs1, rs3, rs2}
\item {\tt fsri rd, rs1, rs3, imm}
\end{itemize}
\item RV64 only:
\begin{itemize}
\item {\tt fslw  rd, rs1, rs3, rs2}
\item {\tt fsrw  rd, rs1, rs3, rs2}
\item {\tt fsriw rd, rs1, rs3, imm}
\end{itemize}
\end{itemize}


\subsubsection{Carryless Multiply}
Carryless multiply is an essential part of Galois/Counter Mode (GCM)
operation for authenticated encryption using block-ciphers.
It is also useful for hashing and as a pseudo-random number generator.

Instruction Listing:
\begin{itemize}
\item RV32, RV64:
\begin{itemize}
\item {\tt clmul  rd, rs1, rs2}
\item {\tt clmulh rd, rs1, rs2}
\item {\tt clmulr rd, rs1, rs2}
\end{itemize}
\item RV64 only:
\begin{itemize}
\item {\tt clmulw  rd, rs1, rs2}
\item {\tt clmulhw rd, rs1, rs2}
\item {\tt clmulrw rd, rs1, rs2}
\end{itemize}
\end{itemize}

\subsubsection{Bitfield Insert/Extract/Permute}
Permutation layers are common constructions in block-ciphers.
Also, RISC-V is a little endian system, and certain block ciphers (SHA2)
require big endian byte-ordering for their inputs.

Instruction Listing:
\begin{itemize}
\item RV32, RV64:
\begin{itemize}
\item {\tt bext   rd, rs1, rs2}
\item {\tt bdep   rd, rs1, rs2}
\item {\tt grev   rd, rs1, rs2}
\item {\tt grevi  rd, rs1, rs2}
\end{itemize}
\item RV64 only:
\begin{itemize}
\item {\tt bextw  rd, rs1, rs2}
\item {\tt bdepw  rd, rs1, rs2}
\item {\tt grevw  rd, rs1, rs2}
\item {\tt greviw rd, rs1, rs2}
\end{itemize}
\end{itemize}


