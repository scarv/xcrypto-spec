% =============================================================================

%
% TODO: FAQ items
%
% - Why a "general purpose" ISA? What does this mean wrt. "special purpose"?
%
%


\begin{description}

\item[Is there any logic behind the \XCRYPTO instruction mnemonics?]
      \XCRYPTO instruction mnemonics follow a (somewhat) consistent scheme; 
      they all include a domain-separating prefix \VERB[RV]{xc}, and, if
      need be, a suffix intended to identify specific variants.  These 
      include
      \[
      \begin{array}{lcl}
      \VERB[RV]{i} &:& \mbox{immediate (vs. register)   }        \\
      \VERB[RV]{u} &:& \mbox{updating  (vs. overwriting)}        \\
      \VERB[RV]{h} &:& \mbox{high}                               \\
      \VERB[RV]{l} &:& \mbox {low}                               \\
      \VERB[RV]{b} &:& \mbox     {byte oriented}                 \\
      \VERB[RV]{h} &:& \mbox{half-word oriented}                 \\
      \VERB[RV]{w} &:& \mbox     {word oriented}                 \\
      \VERB[RV]{x} &:& \mbox{a size, length, or count parameter} \\
      \end{array}
      \]

\item[What restrictions exist wrt. the \XCRYPTO memory interface?]
      Although they share a memory interface, there is no requirement for
      the host core and co-processor to share an address space.  Indeed,
      it is easy to imagine cases where using disjoint address spaces is
      useful; examples include use of
      a) an OTP-style memory for key storage, 
         or
      b) a (uncached) scratch-pad memory,
      by the co-processor alone.


\item[Why include instruction class $2.1$ (RNG), when alternative X would be better?]
      Although still further options may be possible, it seems obvious
      that one {\em could} expose the RNG via

      \begin{itemize}
      \item a set of instructions,
      \item one or more CSRs,
            or
      \item a memory-mapped peripheral.
      \end{itemize}

      \noindent
      Trade-offs wrt. various metrics naturally result.  For example,
      the first option demands space in the instruction encoding, 
      whereas
      the last  option can be accessed using existing memory access instructions but with an impact on latency;
      at best, it seems unclear there is a single ``best'' option.

% --------------
    There are numerous other open questions about how to build a
    randomness interface.
    Some of which apply more to this instantiation, others more generally:

    \begin{description}
    \item[How exactly should the random numbers be generated?]
    This will depend on the exact implementation of the system.  Different
    primitives are available for FPGAs and ASICs, and different technology
    processes may favour different TRNG instantiations over others.  We
    believe that the exact implementation of the T/PRNG should be left to the
    implementor, so long as they are able to realise random number sources
    which fulfill the cryptographic appropriateness requirements.

    \item[What does {\em cryptographically appropriate randomnes} mean?]
    This is short hand for a source of randomness of sufficient quality so as
    to not introduce any weakness into the cryptographic system.  There are
    standard sets of tests which are used to validate this property, as well
    as official guidance from NIST.
    See \cite{SCARV:NIST:SP:800_90a},
        \cite{SCARV:NIST:SP:800_90b} and
        \cite{SCARV:NIST:SP:800_90c}.

    \item[How does one functionally verify such instructions?]
    A TRNG is impossible to model during event-based functional simulation.
    It is even more impossible to create a golden reference model for a TRNG
    against which to compare an implementation.  One strategy for functional
    verification would be to split the functionality into parts. A
    verification engineer might replace the TRNG instance with something
    easily modelled with standard verification flows. This would allow
    verification of the architectural functionality of the instructions.
    Likewise, a formal verification flow could black-box the TRNG and check
    that the interface to it works as expected.  Verification of the
    randomness quality would then need to be done after the device is
    implemented.

    \item[How should post-processing of the TRNG output be handled?]
    Including post-processing and on-line quality checking inside the
    hardware implementation arguably makes things simpler for the programmer.
    It is also possible that some programmers would {\em not trust} the TRNG
    to do post-processing, and want to do it themselves.

    \item[How should access to these instructions be controlled?]
    Shared hardware state represents a possible attack vector.  The
    criticality of randomness to key generation means that these instructions
    deserve special attention in this regard.

    \begin{itemize}

    \item Sharing at a low level might involve storing the state of the PRNG
    during a context switch. The TRNG is harder to manage in this regard.

    \item At a higher level, if there are applications which do not mutually
    trust one another running on the system, it is reasonable to assume there
    is an operating system there to enforce process separation.  Such an
    operating system could ensure a level of access control to the RNG which
    stops accesses from different user-level processes being interleaved.

    \item Alternatively, the instructions could be machine-mode only, and
    access to user-level processes granted via a system call.
    \end{itemize}

    \item[Should there be a throughput requirement for the generators?]
    When building a TRNG or PRNG, there is considerable scope for trading
    latency and throughput of the generators for area and energy consumption.
    It may be sensible to have a platform level parameter to document how
    fast new randomness can be generated. E.g. the target can generate a new
    32-bit random number every $X$us.

    \end{description}

% --------------

\item[What is the {\tt xc.lut} instruction useful for?]
Light-weight block ciphers often make use of small, 4-bit SBox
transformations as part of their substitution layer.
To all intents and purposes, an SBox is just a lookup table.
Such SBoxes can either be implemented in software as a lookup table, or
as a set of transforming instructions with compute the new value each time.
Both of these approaches are relatively slow, and/or energy intensive
compared to the size of the data being operated on.
This instruction allows 4-bit sboxes to be computed entirely
inside the core, in parallel, and allows the operation to be masked against
side-channel attack as well.
For \RVXLEN = 32, 8 SBoxes are computed.
For \RVXLEN = 64, 16 SBox are computed.
The instruction is not specific to any one algorithm.

% --------------

\item[Multi-precison integer instructions]
\label{sec:design:mpn}

Here we present alternative instruction sequences between the proposed
multi-precision instructions and standard RISC-V code.

Note that some of these instructions can have their encodings
made more compact by either destroying their sources, or
implicitly reading their destination registers.

    \begin{description}

    \item[xc.mmul.3] \hfill

        \begin{lstlisting}[frame=single,title=XCRYPTOrypto,numbers=left]
        xc.mmul.3   (rd2,rd1), rs1, rs2, rs3 // Note, rd2 = rd1 | 0x1
        \end{lstlisting}

        \begin{lstlisting}[frame=single, title=RISC-V RV32IM / RV64IM,numbers=left]
        mul   t1 , rs1, rs2
        mulhu rd2, rs1, rs2 // (rd2, t1) = rs1*rs2
        add   rd1, t1 , rs3 //      rd1  = low(rs1*rs2)+rs3
        sltu  t1 , rd1, t1  //      rs2  = rd1 < rs2
        add   rd2, rd2, rs2 //      rd2 += overflow bit?
        \end{lstlisting}

    \item[xc.macc.1] \hfill

        \begin{lstlisting}[frame=single,title=XCRYPTOrypto,numbers=left]
        xc.macc.1 (rd2,rd1), rs1, rs2, rs3 // Note, rd2 = rd1 | 0x1
        \end{lstlisting}

        \begin{lstlisting}[frame=single, title=RISC-V RV32IM / RV64IM,numbers=left]
        add     rd1, rs1, rs3
        sltu    rs1, rd1, rs1
        add     rd2, rs2, rs1
        \end{lstlisting}

    \item[xc.madd.3] \hfill

        \begin{lstlisting}[frame=single,title=XCRYPTOrypto,numbers=left]
        madd3   (rd2,rd1), rs1, rs2, rs3 // Note, rd2 = rd1 | 0x1
        \end{lstlisting}

        \begin{lstlisting}[frame=single, title=RISC-V RV32IM / RV64IM,numbers=left]
        add     rs3, rs3, rs2
        add     rd1, rs1, rs3
        sltu    rd2, rs3, rs2
        sltu    rs3, rd1, rs3
        add     rd2, rd2, rs3
        \end{lstlisting}

    \item[xc.msub3] \hfill

        \begin{lstlisting}[frame=single,title=XCRYPTOrypto,numbers=left]
        msub3   (rd2,rd1), rs1, rs2, rs3 // Note, rd2 = rd1 | 0x1
        \end{lstlisting}

        \begin{lstlisting}[frame=single, title=RISC-V RV32IM / RV64IM,numbers=left]
        sub     rd2, rs1, rs2
        mv       t0, rs1
        sltu     t0,  t0, rd2
        sub     rd1, rd2, rs3
        neg      t0,  t0
        sltu    rd2, rd2, rd1
        sub     rd2, t0 , rd2
        \end{lstlisting}

    \end{description}

% --------------

\item[Why include the special purpose SHA instructions?]
      While \XCRYPTO tries to include only {\em general purpose} instructions and
      avoid overly specific instructions, in some cases it deviates from this:

      \noindent SHA3, and the underlying Keccak round function:
      \begin{itemize}
      \item While the compute elements of Keccak are well supported by other
            \XCRYPTO instructions (and other proposals in the literature), there
            is considerable latent complexity in generating indices into
            the state matrix.
            This complexity is removed by unrolling the loops and
            turning addresses into constant offsets. However, this is not
            always appropriate in memory constrained environments like embedded
            cores.
      \item The index computation relies on integer arithmetic modulo 5. This
            is very slow to implement using the normal RISC-V {\tt mod} instruction.
            Lookup tables can improve performance at significant energy efficiency cost.
      \item The cost of index computation remains the same across all SHA3 parameter
            sets, meaning that for smaller parameters, it can dominate the runtime.
      \item Code-dense implementations of Keccak which are runtime competitive
            with unrolled versions can bring considerable energy-efficiency
            benefits in systems with instruction caches.
      \item Keccak (particularly the CSHAKE instantiation) is used by many of the
            candidate algorithms submitted to the NIST post-quantum public-key
            cryptography competition. This further increases its importance as
            an optimisation target.
      \end{itemize}

      \noindent SHA2 round functions:
      \begin{itemize}
      \item SHA256 is an extremely popular hash function.
      \item Light weight accelerator instructions, while not as efficient
        as {\em full round} instructions found in larger cores, offer
        a very promising trade off between implementation cost and
        performance/code-size/energy-efficiency gains.
      \end{itemize}

      \noindent Both instruction classes:
      \begin{itemize}
      \item The overhead in terms of resources (LUTs or logic cells) is very
            small compared to the benefits the instructions provide.
      \end{itemize}

\end{description}

% =============================================================================
