% =============================================================================

%
% TODO: FAQ items
%
% - Why a "general purpose" ISA? What does this mean wrt. "special purpose"?
%
%


\begin{description}

\item[Is there any logic behind the \XCID instruction mnemonics?]
      \XCID instruction mnemonics follow a (somewhat) consistent scheme; 
      they all include a domain-separating prefix \VERB[RV]{xc}, and, if
      need be, a suffix intended to identify specific variants.  These 
      include
      \[
      \begin{array}{lcl}
      \VERB[RV]{i} &:& \mbox{immediate (vs. register)   }        \\
      \VERB[RV]{u} &:& \mbox{updating  (vs. overwriting)}        \\
      \VERB[RV]{h} &:& \mbox{high}                               \\
      \VERB[RV]{l} &:& \mbox {low}                               \\
      \VERB[RV]{b} &:& \mbox     {byte oriented}                 \\
      \VERB[RV]{h} &:& \mbox{half-word oriented}                 \\
      \VERB[RV]{w} &:& \mbox     {word oriented}                 \\
      \VERB[RV]{x} &:& \mbox{a size, length, or count parameter} \\
      \end{array}
      \]

\item[What restrictions exist wrt. the \XCID memory interface?]
      Although they share a memory interface, there is no requirement for
      the host core and co-processor to share an address space.  Indeed,
      it is easy to imagine cases where using disjoint address spaces is
      useful; examples include use of
      a) an OTP-style memory for key storage, 
         or
      b) a (uncached) scratch-pad memory,
      by the co-processor alone.


\item[Why include instruction class $2.1$ (RNG), when alternative X would be better?]
      Although still further options may be possible, it seems obvious
      that one {\em could} expose the RNG via

      \begin{itemize}
      \item a set of instructions,
      \item one or more CSRs,
            or
      \item a memory-mapped peripheral.
      \end{itemize}

      \noindent
      Trade-offs wrt. various metrics naturally result.  For example,
      the first option demands space in the instruction encoding, 
      whereas
      the last  option can be accessed using existing memory access instructions but with an impact on latency;
      at best, it seems unclear there is a single ``best'' option.

\item[Why include the special purpose class $3.3$ SHA3 instructions?]
      While \XCID tries to include only {\em general purpose} instructions and
      avoid overly specific instructions, in some cases it deviates from this:

      \begin{itemize}
      \item SHA3, and the underlying Keccak round function,
      \item While the compute elements of Keccak are well supported by other
            \XCID instructions (and other proposals in the literature), there
            is considerable latent complexity in generating indices into
            the state matrix.
            This complexity is removed by unrolling the loops and
            turning addresses into constant offsets. However, this is not
            always appropriate in memory constrained environments like embedded
            cores.
      \item The index computation relies on integer arithmetic modulo 5. This
            is very slow to implement using the normal RISC-V {\tt mod} instruction.
            Lookup tables can improve performance at significant energy efficiency cost.
      \item The cost of index computation remains the same across all SHA3 parameter
            sets, meaning that for smaller parameters, it can dominate the runtime.
      \item Code-dense implementations of Keccak which are runtime competitive
            with unrolled versions can bring considerable energy-efficiency
            benefits in systems with instruction caches.
      \item Keccak (particularly the CSHAKE instantiation) is used by many of the
            candidate algorithms submitted to the NIST post-quantum public-key
            cryptography competition. This further increases its importance as
            an optimisation target.
      \item The overhead in terms of resources (LUTs or logic cells) is very
            small compared to the benefits the instructions provide.
      \end{itemize}

\end{description}

% =============================================================================
