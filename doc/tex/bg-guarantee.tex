% =============================================================================

\subsection{Guarantees}
\label{sec:bg:guarantee}

\begin{itemize}

\item A raft of challenges stemming from partially defined behaviour in the
      C programming language, and associated compilers, is well documented
      (see, e.g.,~\cite[Section 2.1]{SCARV:SimChiAnd:18}).
      These are particularly problematic for high-assurance software, where
      precise control over all aspects of execution behaviour forms a basis
      for guarantees wrt. security
      (see, e.g., various counterexamples such as~\cite{SCARV:KPVV:16}).
      A similar argument can obviously be made in relation to the platform 
      on which resulting software is executed.

      The specification of \XCID guarantees to exclude all
                   undefined 
      and
      implementation defined
      functionality or semantics:
      this is intended to offer the greatest possible control over and thus
      transparency wrt. execution behaviour.

\item The assumption of a shared memory interface, plus the lack of control
      over components beyond that interface 
      (e.g., lower-level caches) 
      limits the extent to which general guarantees wrt. execution latency 
      can be made.  However, an implementation of \XCID is required to 
      a) guarantee that all computational 
         (i.e., {\em excluding} memory access) 
         instructions will exhibit constant (or data-oblivious) execution 
         latency,
         and
      b) clearly quote said constant in associated documentation.

      A similar concept was mooted for introduction in ARMv8.4-A, using a
      processor mode to enforce data-independent execution latency for the
      same class of instructions; the concept is motivated, for example,
      by the ARM IoT Security Manifesto\footnote{
      \url{http://pages.arm.com/iot-security-manifesto.html}
      } (see, e.g., Page $10$ re. ``the rise of side-channels '').
      However, it seems unclear\footnote{
      \url{http://twitter.com/Kensan42/status/946681596001902593}
      } {\em why} the original\footnote{
      \url{http://web.archive.org/web/20171108010153/http://community.arm.com/processors/b/blog/posts/introducing-2017s-extensions-to-the-arm-architecture}
      } announcement was later amended to remove this feature\footnote{
      \url{http://community.arm.com/processors/b/blog/posts/introducing-2017s-extensions-to-the-arm-architecture}
      }.

\end{itemize}

% =============================================================================
