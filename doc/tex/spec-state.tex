\subsection{\XCID state}
\label{sec:spec:state}

It is important to recognise the overhead, wrt. both 
time  (e.g., due to it needing to be context switched) 
and 
space (e.g., wrt. additional logic),
relating to {\em any} state added by \XCID.
However, it also seems reasonable to align the trade-off(s) involved with 
existing, common cases such as ISEs for floating-point arithmetic (namely
the standard 
F~\cite[Section 8]{SCARV:RV:ISA:I:17}
and
D~\cite[Section 9]{SCARV:RV:ISA:I:17}
extensions).  The addition of a dedicated floating-point register file is
rationalised, for example, by
a) clear separation of duty,
b) additional capacity, leading, e.g., to enhanced performance,
   and
c) additional (specialised) functionality;
a similar argument is true of \XCID, but motivated by a different context
(i.e., cryptographic vs. floating-point workloads).

Note that, 
particularly due to the nature of workloads supported by \XCID,
challenges relating to management of additional state are important.  For
example, attack vectors such as 
LazyFP~\cite{SCARV:StePre:18}, 
which capitalises on a short-cut wrt. the overhead of context switching
x87 state, {\em must} be robustly mitigated.

% =============================================================================

\subsubsection{$\XCR$ register file}
\label{sec:spec:state:xcr}

\XCID 
requires one additional 
$16$-element, $\XCLEN$-bit register file.
We refer to this as the $\XCR$ register file, distinguishing it from the
general-purpose $\GPR$ register file specified by the base ISA.
Note that

\begin{itemize}
\item Upon reset of \XCID,
            the $\XCR$ register file {\em must} be reset (cf.~\REFSEC{sec:spec:instruction:xc.init}).
      st. $\XCR[*][i] = 0$ for $0 \leq i < |\XCR|$.
\item RISC-V specifies~\cite[Section 2.1]{SCARV:RV:ISA:I:17} that
      $
      \GPR[*][0] = 0 ,
      $
      i.e., that the $0$-th $\GPR$ register is fixed to $0$: any read from 
      said register yields $0$.  The analogous special-case is {\em not} 
      true of 
      $
      \XCR[*][0] ,
      $
      the $0$-th $\XCR$ register.
\item The $\XCR$ register file is used {\em exclusively} by \XCID, but, on
      the other hand, \XCID can {\em also} use the $\GPR$ register file in 
      selected cases.  For example, it uses a $\GPR$ register as

      \begin{itemize}
      \item a source, specifically a base address, for memory accesses,
            (e.g., \VERB[RV]{xc.ld.w} and \VERB[RV]{xc.st.w}; see \REFSEC{sec:spec:instruction:xc.ld.w} and \REFSEC{sec:spec:instruction:xc.st.w}),
      \item a destination for comparison operations,
            (e.g., \VERB[RV]{xc.mequ}; see \REFSEC{sec:spec:instruction:xc.mequ}).
      \end{itemize}

      \noindent
      Doing so aligns with \REFSEC{sec:bg:concept}, where the host core is 
      pitched as a control-path for the co-processor: address computation,
      control-flow orchestration, etc. fall under the remit of the former, 
      since they can be supported by the base ISA as is.
\end{itemize}    

% =============================================================================

\subsubsection{Control and Status Registers (CSRs)}
\label{sec:spec:state:csr}

\begin{table}[p]
\begin{center}
\begin{tabular}{|ccc|}
\hline
Name          & Address         & Access     \\ 
\hline
$\SPR{mxcsr}$ & \RADIX{7C0}{16} & read/write \\
$\SPR{sxcsr}$ & \RADIX{DC0}{16} & read/write \\
$\SPR{uxcsr}$ & \RADIX{CC0}{16} & read       \\
\hline
\end{tabular}
\end{center}
\caption{A summary of \XCID CSRs, per~\cite[Table 2.1]{SCARV:RV:ISA:II:17}.}
\label{tab:csr}
\end{table}

\begin{figure}[p]
\begin{center}
\begin{bytefield}[bitwidth={1.4em},bitheight={8.0ex},endianness=big]{32}
\bitheader{0-31}               
\\
  \bitbox{15}{\rule{\width}{\height}}
& \bitbox{ 1}{\rule{\width}{\height}}
& \bitbox{ 1}{\rotatebox{90}{\tiny $\ID{SHA3}     $}}
& \bitbox{ 1}{\rotatebox{90}{\tiny $\ID{AES }     $}}
& \bitbox{ 1}{\rotatebox{90}{\tiny $\ID{PACK}_{ 2}$}}
& \bitbox{ 1}{\rotatebox{90}{\tiny $\ID{PACK}_{ 4}$}}
& \bitbox{ 1}{\rotatebox{90}{\tiny $\ID{PACK}_{ 8}$}}
& \bitbox{ 1}{\rotatebox{90}{\tiny $\ID{PACK}_{16}$}}
& \bitbox{ 1}{\rotatebox{90}{\tiny $\ID{PACK}_{32}$}}
& \bitbox{ 1}{\rotatebox{90}{\tiny $\ID{MP  }     $}}
& \bitbox{ 1}{\rotatebox{90}{\tiny $\ID{BIT }     $}}
& \bitbox{ 1}{\rotatebox{90}{\tiny $\ID{MEM }     $}}
& \bitbox{ 1}{\rotatebox{90}{\tiny $\ID{RNG }     $}}
& \bitbox{ 2}{\rotatebox{90}{\tiny $\ID{XS  }     $}}
& \bitbox{ 1}{\rule{\width}{\height}}
& \bitbox{ 1}{\rotatebox{90}{\tiny $\ID{S   }     $}}
& \bitbox{ 1}{\rotatebox{90}{\tiny $\ID{U   }     $}}
\\
\end{bytefield}
\end{center}
\caption{A diagrammatic description of the $\SPR{mxcsr}$ register; blanked regions are reserved, so yield zero when the register is read.
Note that we force implementations to yield zero, rather than requiring software to simply ignore these bits
(as happens in the base RISC-V specification) to remove any possibility of implementation specific behaviour.}
\label{fig:mxcsr}
\end{figure}

\begin{table}[p]
\begin{center}
\begin{tabular}{|l|c|c|l|}
\hline
Field             & Index          & Access & Description                                                                         \\ 
\hline
$\ID{U   }      $ & $           0$ & WLRL   & Indicates whether or not \XCID is accessible in user mode                           \\
$\ID{S   }      $ & $           1$ & WLRL   & Indicates whether or not \XCID is accessible in supervisor mode                     \\
$\ID{XS  }      $ & $ 4 \RANGE  3$ & R/O    & Indicates whether or not the \XCID state is dirty                                   \\
$\ID{RNG }      $ & $           5$ & R/O    & Is RNG  feature class, per \REFSEC{sec:bg:feature}, supported (set), or not (clear) \\
$\ID{MEM }      $ & $           6$ & R/O    & Is MEM  feature class, per \REFSEC{sec:bg:feature}, supported (set), or not (clear) \\
$\ID{BIT }      $ & $           7$ & R/O    & Is BIT  feature class, per \REFSEC{sec:bg:feature}, supported (set), or not (clear) \\
$\ID{MP  }      $ & $           8$ & R/O    & Is MP   feature class, per \REFSEC{sec:bg:feature}, supported (set), or not (clear) \\
$\ID{PACK}_{32} $ & $           9$ & R/O    & Are operations on sub-word  of $w = 32$ bits supported (set), or not (clear)        \\
$\ID{PACK}_{16} $ & $          10$ & R/O    & Are operations on sub-words of $w = 16$ bits supported (set), or not (clear)        \\
$\ID{PACK}_{ 8} $ & $          11$ & R/O    & Are operations on sub-words of $w =  8$ bits supported (set), or not (clear)        \\
$\ID{PACK}_{ 4} $ & $          12$ & R/O    & Are operations on sub-words of $w =  4$ bits supported (set), or not (clear)        \\
$\ID{PACK}_{ 2} $ & $          13$ & R/O    & Are operations on sub-words of $w =  2$ bits supported (set), or not (clear)        \\
$\ID{AES }      $ & $          14$ & R/O    & Is AES  feature class, per \REFSEC{sec:bg:feature}, supported (set), or not (clear) \\
$\ID{SHA3}      $ & $          15$ & R/O    & Is SHA3 feature class, per \REFSEC{sec:bg:feature}, supported (set), or not (clear) \\ 
\hline
\end{tabular}
\end{center}
\caption{A tabular description of the $\SPR{mxcsr}$ register.}
\label{tab:mxcsr}
\end{table}

\XCID 
requires three additional
non-standard CSRs~\cite[Section 2]{SCARV:RV:ISA:II:17},
outlined by \REFTAB{tab:csr}.
In practice, only {\em one} additional register is required for
$\SPR{mxcsr}$; $\SPR{sxcsr}$ and $\SPR{uxcsr}$ are shadowed versions of said register, accessible
within specific modes.

\paragraph{The $\SPR{mxcsr}$ register}

The
$\SPR{mxcsr}$
register is a machine mode CSR is responsible for
 inspecting (e.g., feature identification)
and 
controlling (e.g., supervisor and user mode access)
\XCID.
In combination, \REFTAB{tab:mxcsr} and \REFFIG{fig:mxcsr} provide an
overview of the register.

\begin{itemize}

\item The 
      $\SPR{mxcsr}[*][\ID{U}]$
      and
      $\SPR{mxcsr}[*][\ID{S}]$
      fields control \XCID wrt. user and supervisor mode respectively.  
      Imagine an \XCID instruction is executed in 
      user                         (resp. supervisor) 
      mode: if 
      $\SPR{mxcsr}[*][\ID{U}] = 1$ (resp. $\SPR{mxcsr}[*][\ID{S}] = 1$)
      the instruction executes as normal, otherwise, if
      $\SPR{mxcsr}[*][\ID{U}] = 0$ (resp. $\SPR{mxcsr}[*][\ID{S}] = 0$),
      an illegal opcode exception results.
      If 
      $\SPR{mxcsr}[*][\ID{U}] = 1$
      then is {\em must} be true that
      $\SPR{mxcsr}[*][\ID{S}] = 1$;
      if the fields are updated st.
      $\SPR{mxcsr}[*][\ID{U}] = 1$
      and
      $\SPR{mxcsr}[*][\ID{U}] = 1$,
      an illegal opcode exception results.

\item The 
      $\SPR{mxcsr}[*][\ID{XS}]$
      field indicates whether or not the \XCID state is dirty (i.e., it
      has been updated), and must be preserved during a context switch;
      this field is analogous to the 
      $\SPR{mstatus}[*][\ID{FS}]$
      field described in~\cite[Section 3.1.6.5]{SCARV:RV:ISA:II:17}.
      More specifically,
      \[
      \begin{array}{l@{\;}c@{\;}rcl}
      \SPR{mxcsr}[*][\ID{XS}] &=& \RADIX{00}{2} &\rightarrow& \mbox{\XCID is disabled                                                                  } \\
      \SPR{mxcsr}[*][\ID{XS}] &=& \RADIX{01}{2} &\rightarrow& \mbox{\XCID is  enabled, state is initialised (i.e.,            values are per reset)    } \\
      \SPR{mxcsr}[*][\ID{XS}] &=& \RADIX{10}{2} &\rightarrow& \mbox{\XCID is  enabled; state is clean       (i.e., {\em   no} values have been updated)} \\
      \SPR{mxcsr}[*][\ID{XS}] &=& \RADIX{11}{2} &\rightarrow& \mbox{\XCID is  enabled; state is dirty       (i.e., {\em some} values have been updated)} \\
      \end{array}
      \]
      noting that

      \begin{itemize}
      \item the encodings for
            $\SPR{mxcsr}[*][\ID{XS}]$
            mirror those for 
            $\SPR{mstatus}[*][\ID{FS}]$,
            e.g., per~\cite[Section 3.1.6.5]{SCARV:RV:ISA:II:17},
      \item where a change from
            $\SPR{mxcsr}[*][\ID{XS}] = \RADIX{00}{2}$
            to
            $\SPR{mxcsr}[*][\ID{XS}] = \RADIX{01}{2}$
            occurs, 
            the $\XCR$ register file {\em must} be reset (cf.~\REFSEC{sec:spec:instruction:xc.init}),
      \item $\SPR{mxcsr}[*][\ID{XS}]$ is deliberately located toward the
            least-significant end of $\SPR{mxcsr}$; this allows atomic 
            access via one \VERB[RV]{csrrwi} instruction, for example,
      \item in the case where
            $\SPR{mxcsr}[*][\ID{XS}] = \RADIX{00}{2}$,
            {\em all} \XCID instructions (even when executed in machine
            mode) result in an illegal opcode exception; since \XCID is
            essentially disabled, this fact {\em may} be harnessed, for 
            example, to realise a form of power saving optimisation.
      \end{itemize}            

\end{itemize}

\paragraph{The $\SPR{uxcsr}$ and $\SPR{sxcsr}$ registers}

The 
$\SPR{uxcsr}$ 
and 
$\SPR{sxcsr}$ 
registers are user and supervisor mode CSRs respectively, acting as
shadowed versions of the machine mode $\SPR{mxcsr}$.  As such, their
definition and semantics are identically, bar the following caveats:

\begin{itemize}
\item If
      $\SPR{mxcsr}[*][\ID{S}] = 0$,
      then
      $\SPR{uxcsr}$ 
      and 
      $\SPR{sxcsr}$ 
      yield  zero when read, 
      and result  in an illegal opcode exception when written.
\item If
      $\SPR{mxcsr}[*][\ID{U}] = 0$,
      then
      $\SPR{uxcsr}$ 
      yields zero when read, 
      and results in an illegal opcode exception when written.
\end{itemize}

% =============================================================================

\subsubsection{Application Binary Interface (ABI)}
\label{sec:spec:state:abi}

\begin{itemize}
\item For the purposes of the ABI, {\em all} additional state stemming from 
      \XCID is considered to be callee-save: if function $f$ calls function
      $g$, then, for example, $g$ is deemed responsible for using the stack
      to preserve (resp. restore) any content in the $\XCR$ register file 
      it destroys during execution.
\item The first eight registers, 
      i.e., $\XCR[*][0]$ through $\XCR[*][7]$, 
      are considered to be function arguments (including return values).
      All other registers, 
      i.e., $\XCR[*][8]$ upward,
      are considered to be temporaries.
\item Any $\XCR$ registers pushed to the stack should be stored 
      at the end of the stack frame, 
      i.e., {\em after} content associated with base ISA (plus any standard 
      extensions thereof, e.g., floating-point content).
\end{itemize}

% =============================================================================
