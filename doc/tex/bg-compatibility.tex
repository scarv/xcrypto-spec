% =============================================================================

\subsection{Compatibility}
\label{sec:bg:compatibility}

\begin{itemize}

\item Per the current list\footnote{
      See, e.g., \url{http://workspace.riscv.org}
      } of (public) RISC-V working groups, 
      \XCID relates to elements of (at least) the following:

      \begin{enumerate}
      \item Cryptographic Extensions Task Group,
      \item BitManip\footnote{
            \url{https://github.com/riscv/riscv-bitmanip}
            }                             Group,
            and
      \item P             Extension  Task Group (i.e., the embedded DSP-like ISE).
      \end{enumerate}

      \noindent
      In some cases overlap exists, potentially representing an opportunity
      for unification; in other cases the underlying goal, approach, and/or
      ethos is distinct and thus incompatible.

\item Several features of \XCID are intended to avoid conflicts with existing 
      or future extensions:

      \begin{itemize}
      \item the instruction encoding is constrained to the 
            \RVCUSTOM{1}~\cite[Table 19.1]{SCARV:RV:ISA:I:17}
            space, i.e., a $25$-bit space with prefix $\RADIX{0101011}{2}$,
      \item all instruction mnemonics include a domain-separating prefix,
            meaning, for example, \VERB[RV]{foo} becomes \VERB[RV]{xc.foo}.
      \end{itemize}
      
      \noindent
      Note that these should be considered flexible placeholders: they can
      be changed to suit future constraints with little or no impact.

\item \XCID is intended to be agnostic wrt. implementation for a given host 
      core.  As such, there is no conceptual reason preventing use within
      RV64I or RV128I (vs. RV32I as is notionally expected).  

\end{itemize}

% =============================================================================
