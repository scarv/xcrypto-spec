% =============================================================================

Versus a more general case, cryptographic workloads are a challenge in the
sense they typically
a) require computationally intensive, somewhat niche functionality,
   and
b) form a central target in what is a complex, evolving attack surface.
The former is a particular issue, because cryptography normally represents
an enabling technology vs. a feature: put another way, it will represent
overhead when viewed from the perspective of a user.  Efficiency is hence 
a goal in and of itself, but {\em also} as an enabler for security.  This 
is because one cannot (or at least should not) compromise security to meet 
efficiency requirements, so delivering higher efficiency can be pitched as 
an enabler for countermeasures against attack (since there will be more 
margin within which to do so).

This document acts as the specification for a 
greenfield, non-standard extension~\cite[Section 21.1]{SCARV:RV:ISA:I:17} 
to any one the RISC-V base ISAs
(e.g., RV32I~\cite[Section 2]{SCARV:RV:ISA:I:17}, RV32E~\cite[Section 3]{SCARV:RV:ISA:I:17}, RV64I~\cite[Section 4]{SCARV:RV:ISA:I:17}, or RV128I~\cite[Section 5]{SCARV:RV:ISA:I:17}),
which we dub \XCID; it forms an output from the SCARV\footnote{
\url{http://www.scarv.org}, \url{http://www.github.com/scarv}
} project, funded by EPSRC\footnote{
\url{http://gow.epsrc.ac.uk/NGBOViewGrant.aspx?GrantRef=EP/R012288/1}
} as part of the UK-based RISE\footnote{
\url{http://www.ukrise.org}
} initiative.  
\XCID aims to enable
a) efficient
   and
b) secure
software implementation of cryptographic primitives, within a remit which
is conceptually analogous to that of a floating-point co-processor.

Note this document {\em is} a specification for \XCID, but categorically
{\em is not} an implementation guide: we provide a separate document for 
that purpose.  In order to avoid the specification becoming too verbose,
we defer a detailed description of the notation used, related work, and 
design notes to
\REFAPPX{appx:notation},
\REFAPPX{appx:related},
and
\REFAPPX{appx:design}
respectively.  The document, and project as a whole, follows a semantic 
versioning\footnote{
\url{http://semver.org}
} (or major/minor/patch) convention, with a changelog\footnote{
\url{http://keepachangelog.com}
} maintained within the repository (vs. in the document).
In line with the current version, the specification should be viewed as an
initial prototype or draft; statements such as 
``\XCID is   X'' 
or
``\XCID does Y''
should be carefully qualified with {\em currently} vs. {\em definitively}.  
In particular, we expect some degree of iteration and so change to emerge 
from work in progress wrt. design, implementation, and evaluation.

% =============================================================================
